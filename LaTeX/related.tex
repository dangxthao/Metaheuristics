\section{Related Work and Conclusion}
Related work in falsification is already reviewed in the introduction, we present here a brief discussion of related work in the idea of metaheuristics combination. Our approach is inspired by the hybrid metaheuristics resulting from a hybridization of different metaheuristics, or hybridization of metaheuristics with some heuristics (see the survey \cite{Talbi2009}, the books \cite{Lones2011,Talbi2013,Talbi2014} and abundant references therein). The overall goal of such hybridizations is to combine the strengths of exploitation and exploration. A hybridization can be at a low level, that is component level. More concretely, a component of a metaheuristics is replaced by a metaheuristic. For example, inside an Evolutionary Algorithm the mutation operator can be performed by a local search algorithm. A hybridization can also be at a higher level, that is the internal components of the metaheuristics are unchanged, but the metaheuristics are executed sequentially, each using the results of the previous ones. One example is when an exploitation-driven search is run for each individual inside a population-based method in order to improve the population. Our approach indeed belongs to this high level hybridization category; however, in most the existing hybrid algorithms, the hybridization is not adaptive to the current search situation, which is in contrast to our approach. The reason is that in these approaches the hybridization aims only at improving some functionalities or elements of each metaheuristic, or sequential execution of metaheuristics is predefined based only on a fixed time limit for each execution or on exploitation performance (that is, whether the cost value is improved). For example, an exploitation-driven, such as Hill Climbing or Simulated Annealing can have a random restart after running for some fixed amount of time. Our approach additionally uses a global assessment of exploration performance, in terms of coverage measures, to detect important qualitative behavior of the search (in particular entering a basin of attraction around a local optimum) and to make decision on which metaheuristic to execute next. Our notion of coverage is close in spirit to the notion of diversity in population-based methods to determine how similar individuals are. This notion is used in scatter search \cite{Glover2011} to reduce the chance of considering individuals that are too close to each other in some metric. In evolutionary algorithms, diversity is also used in fitness sharing techniques o add a penalty on similarity in the fitness function, or in crowding techniques to select individuals to eliminate or to be replaced by their fitter children \cite{WongWMPZ2012}. We note that while diversity metrics are used in internal functioning of these methods, our coverage measure is used in a higher level to monitor the overall search progress.
