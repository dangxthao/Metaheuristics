\paragraph{Coverage of parameter spaces}	

The metric that we use to measure the coverage of parameter spaces is called {\em cell occupancy}. 

%, which has the following desirable properties:
%\begin{itemize}
%\item The measure is \emph{monotonic}, in the sense that it is guaranteed not to decrease in value when new signals are added to an existing set; 
%\item The measure permits computation with \emph{efficient algorithms};
%\item The measure provides numbers in \emph{reasonable ranges}, in the
%  sense that, for both low dimension and high dimension problems, the
%  measure results in values that are neither too large nor too small
%  so as to be accurately represented with standard floating point numbers.
%\end{itemize}
%
%Henceforth, we define a measure called \emph{cell occupancy} as
%follows.  

Note that our set of signals corresponds
to a set of parameter vectors $\setofcppoints$. We call elements of
$\setofcppoints$ points. %We use $\numpoints$ to denote the size of
%sets $\setofsignals$ and $\setofcppoints$.

Choose a partition of $\cpset$, $\omega=\{ \omega_i | i=1,\ldots,
\numpartition \}$. For now, we assume that each partition element,
which we call a \emph{cell}, is rectangular, with each side of equal
length, $\Delta$, called \emph{grid cell size}\footnote{We note that
  in the setting in which we intend to apply this coverage
  metric, we will expect to select points in $\cpset$ that are no
  closer than some $\epsilon$ distance from each other, based on some
  metric between signals, but this rectangularity will not be exploited in
  the following. Further, we assume that $\epsilon\ll\Delta$.}. A
vector that indicates how many points are in each cell is called a
\emph{distribution}, $\distribution=(n_1,\ldots,n_\numpartition)$,
where each $n_i$ indicates how many points are located in cell $i$.
Cell occupancy is based on the relative number of cells occupied by
points, compared to the total number of cells. Consider the total
number of occupied cells, that is, the number of cells that contain
at least one point, i.e., $\occupiedcellcount =  \sum_{i=1}^{\numpartition} g_i$  
where $g_i = 1$ if  $n_i\geq 1$, and $g_i = 0$ otherwise. Then, the proposed cell occupancy measure is given as
\begin{eqnarray*}
\celloccupancy(\distribution) & = & \frac{\log \occupiedcellcount}{\log \numpartition}.
\end{eqnarray*}
Logarithm functions are used due to the fact that the total number of cells could be very large as compared to the number of occupied cells. The logarithms provide two key features for the cell occupancy measure: (1) they maintain the monotonicity of the measure, and (2) they result in reasonable measure values even for cases where the dimension $\cpdim$ is  large.  
