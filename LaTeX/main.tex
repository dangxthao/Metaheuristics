\documentclass[10pt,oneside,letterpaper]{article}
\usepackage[a4paper, total={6in, 8in}]{geometry}
\usepackage{verbatim}
\usepackage{outlines} % For outlines
\usepackage{enumitem} 
\usepackage{algorithm}
\usepackage{algpseudocode}
\usepackage{outlines,color,url}
\usepackage[colorlinks=true, allcolors=blue]{hyperref}

\usepackage{amsmath,amsthm, amssymb,eufrak}

% For outlines
\setenumerate[1]{label=\Roman*.}
\setenumerate[2]{label=\Alph*.}
\setenumerate[3]{label=\roman*.}
\setenumerate[4]{label=\alph*.}

\newcommand{\jim}[1]{{{\color{blue} \textbf{Jim: #1}}}}

\newcommand{\commentout}[1]{}

\newtheorem{thm}{Theorem}[section]
\newtheorem{defn}{Definition}[section]

\DeclareMathOperator{\sign}{sgn}
\DeclareMathOperator*{\argmin}{arg\,min}
\newcommand{\real}{\mathbb{R}}
\newcommand{\vect}[1]{\mathbf{#1}}
\newcommand{\vx}{\vect{x}}


\newcommand{\solverset}{\mathcal{S}}
\newcommand{\solver}{s}
\newcommand{\param}{p}
\newcommand{\paramset}{P}
\newcommand{\config}{c}
\newcommand{\bestobj}{\rho}
\newcommand{\exectime}{T}



\newcommand{\transfunc}{\Gamma}
\newcommand{\behaviorfunc}{\Phi}
\newcommand{\outsig}{y}

\newcommand{\initcond}{X_0}
\newcommand{\paramsorig}{v}
\newcommand{\inputs}{u}

\newcommand{\outsigset}{\mathcal{Y}}
\newcommand{\outspace}{Y}
\newcommand{\inputsigset}{\mathcal{U}}
\newcommand{\inputspace}{U}
\newcommand{\timeinterval}{\mathcal{I}}
\newcommand{\inputparams}{\hat{u}}
\newcommand{\inputparamset}{\hat{\mathcal{U}}}
\newcommand{\inputgenerator}{g}

\newcommand{\stateset}{\mathcal{X}}
\newcommand{\paramsorigset}{\mathcal{V}}
\newcommand{\inputset}{\mathcal{U}}



% STL
\newcommand{\F}{\Diamond}
\newcommand{\G}{\Box}
\newcommand{\U}{\mathcal{U}}
\newcommand{\spec}{\varphi}
\newcommand{\true}{\top}
\newcommand{\robparam}{\widehat{\rho}}


\title{Metaheuristics to falsify properties of cyber-physical systems}
\author{Thao Dang, James Kapinski, Hisahiro Ito}
\date{}

\begin{document}

\maketitle

\begin{abstract}
Cyber-physical systems (CPSs) are used in many mission-critical applications, and the scale and complexity of these systems is growing rapidly.
New approaches are needed to increase the confidence in the correctness of complex CPS designs.
Falsification techniques offer a practical solution. 
A falsification method takes a system model, along with a specification that defines the correct behavior of the system, and performs a best-effort search over inputs and system parameters for behaviors that violate the specification; essentially performing automated bug finding.
Falsification methods rely on a global optimizer to guide the search; one challenge with falsification is that it can be difficult to select the most effective optimizer and optimization parameters.
To address this, we introduce a new metaheuristic that automatically focuses computing resources on the most effective optimizer for a given falsification task.
Given a collection optimizers, our approach makes heuristic decisions to switch optimizers based on
evolving measures of both cost and the coverage of the decision space.
The technique is able to automatically falsify properties for complex CPS design models more efficiently than existing techniques.
We demonstrate the efficacy of the approach using several examples, including an automotive hydrogen fuel cell control system.

\end{abstract}

\begin{outline}[enumerate]
\1 Introduction
	\2 The challenge of falsification
	\2 Falsification generally and via STL for CPS
	\2 Coverage
	\2 Metaheuristics in optimization
	\2 Overview of our approach
\1 Preliminaries: \emph{For much of the notation, we can reuse the notation from the CAV paper with Arvind, as it provides a way to introduce STL, robustness, and falsification but then quickly move to a general global optimization setting.}
	\2 General notation
		\3 $\Pi$ is a set of global optimization solvers
		\3 For a given solver $\pi\in \Pi$, $\Theta$ is a set of solver \emph{configurations}
		\3 $J$ is a cost evaluation
		\3 $x(t)$ is a trace of the system
	\2 Falsification
	\2 STL
		\3 $\varphi$ is an STL property 
		\3 $\rho$ maps a trace and an STL formula to a real number
	\2 Coverage
		\3 $H$ takes a collection of points and maps it to a real number representing the \emph{coverage} of the space points.
\1 Method
	\2 Monitoring robustness and coverage across iterations
	\2 Changing solver based on heuristic rules, informed by robustness and coverage
	
	
	
	\2 F-Race for tuning parameters
\1 Experiments
\1 Conclusions
\end{outline}


\section{Preliminaries}

We consider dynamical system models, represented by the following mapping

\begin{eqnarray}
\outsig = \transfunc (\paramsorig, \inputs),
\end{eqnarray}
where $\paramsorig \in \paramsorigset$ is a set of parameters that affect the system behaviors, and $\inputs \in \inputset$ is a function of time that represents the inputs to the system.
Parameters $\paramsorig$ could contain a set of system initial conditions as well as some finite set of variables that affect how the system maps inputs to outputs.
Each $\inputs$ is a function $\timeinterval \mapsto \inputspace$, where $\timeinterval$ is an interval (either continuous or discrete) from $0$ to some finite value, and $\inputspace$ is some metric space of finite dimension.
Similarly, we assume that each output signal $\outsig \in \outsigset$ is a function $\timeinterval \mapsto \outspace$, where $\outspace$ is some metric space of finite dimension.

Input $\inputs$ is generally taken from an infinite-dimensional signal space (i.e., these can be partial functions over a continuous time-domain), but we restrict our investigation to the class of  input signals that are finitely parameterizable. That is, we assume that any input signal $\inputs$ 
can be uniquely characterized by a set of $m$ parameters, whose valuation $\inputparams =(\inputparams_1,\ldots,\inputparams_m) \in \inputparamset$ is in a subset of an $m$-dimensional metric space. For example, a right-continuous piecewise constant input signal $\inputs:\timeinterval \rightarrow \real$, where $\timeinterval=[ 0,T ]$, with discontinuities occurring at monotonically increasing instants $\tau_1,\ldots, \tau_m$, where $0=\tau_1<\tau_m<T$, can be uniquely characterized by $m$ values $\inputs(\tau_i)$, and we can define $\inputparams_i=\inputs(\tau_i)$. As another example, a piecewise linear signal over the same set of $m$ discontinuity time points can be also uniquely characterized by $m$ values $\inputparams_i = \inputs(\tau_i)$, that is $t \in [\tau_i, \tau_{i+1}) \;  \inputs(t) = \inputparams_i  + \frac{t -  \tau_i}{\inputparams_{i+1} - \inputparams_i}$.

Let $\inputgenerator:\inputparamset \rightarrow \inputset$ be the function that maps input parameters to input functions. We call elements of $\inputparamset$ parameter points. We define an augmented set of parameters $\param = (\paramsorig, \inputparams)$, where $p\in \paramset = \paramsorigset \times \inputparamset$.
We define a function 
\begin{eqnarray} \label{eq:behaviorfunc}
y &=& \behaviorfunc(\param),
\end{eqnarray}
where $\behaviorfunc(\param) = \transfunc (\paramsorig, \inputgenerator(\inputparams))$.
Note that $\inputgenerator$ is absorbed into the definition of $\behaviorfunc$.

In the previous work, the time intervals are fixed and we search for the signal values at the extreme time points of the intervals. In this work, we extend this framework to classes of piecewise input signals which are more general in two aspects. First, the time intervals can be varied and thus become part of the search space. Second, the signals should satisfy a given temporal constraints. 

%More concretely, in the previous work, we fix the time intervals and search for the signal values for each time interval. Now, using the methods for uniform random and low-discrepancy generation of timed words of timed automata, we need not fix the time intervals which become variables of the underlying optimization problem. 

The motivation for considering the second aspect is to reduce the search space for better efficiency. Indeed, if the specification involves some temporal constraints on the input signals, focusing on such signals would increase the falsification efficiency. 

On the other hand, compared to the previous work, we use here a notion of coverage of temporal specifications. Intuitively it measures the portion of the "good" behaviours satisfying the specification that have been explored.


Let us now use a timed automaton $\mathcal{A}$ that describes the temporal constraints that the input signals should satisfy (a STL formalism can be used as well, but for simplicity of explanation. We know how to generate a good sample of $N$ timed words of length $n$, each of which is of the form $\gamma = (\delta_1, t_1), \ldots, (\delta_n, t_m)$ where $\delta_i$ are labels of discrete transitions. Each label $\delta_i$ corresponds to a range of the real-valued signal values $\param$, or more generally to a constraint $g_i(\param, t) \le 0$ for $t \in [t_i, t_{i+1})$ and $\param \in \paramset$. For piecewise constant signals, these constraints are simply interval constraints. For piecewise linear signals, $g_i$ are linear constraints on $\param$ and $t$. 

Hence, given a timed word $\gamma = (\delta_1, t_1), \ldots, (\delta_n, t_n)$, a real-valued input signal corresponding to $\gamma$ satisfies the following constraints, denoted by $C_{\gamma}(\inputs)$:
$$\forall i \in \{1, \ldots, m \}: \inputs (t)= \param, t \in [t_i, t_{i+1}), g_i(\param,t)  \le 0.$$
And we denote this by $\inputs \models C_{\gamma}(\inputs)$.

\paragraph{Signal Temporal Logic} 

We assume that the correct or expected behaviors for system (\ref{eq:behaviorfunc}) is provided in an unambiguous form that can be efficiently measured and quantified. For this purpose, we use the signal temporal logic (STL) language to define the system specifications.
STL is a modal logic that is well-defined over discrete or real-valued signals and discrete or continuous time \cite{MalerN04}.
STL is appropriate for specifying correct behavior for CPSs, as it can be defined over the real-valued, continuous-time signals that characterize CPS behaviors.
  
Below we present an overview of STL and refer the reader to \cite{MalerN04} for a detailed presentation.
An STL formula $\spec$ consists of atomic predicates along with logical and temporal connectives.
Atomic predicates are defined over signal values and have the form $\spec$, where $f$ is a scalar-valued function over the signal $y$ evaluated at time $t$, and $\sim \in \{ <,\leq, >, \geq, =, \neq \}$.
Temporal operators ``always'' ($\G$), ``eventually'' ($\F$), and ``until'' ($\U$) have the usual meaning and are scoped using intervals of the form $(a,b)$, $(a,b]$, $[a,b)$, $[a,b]$, or $(a,\infty)$, where 
$a,b\in \real_{\geq 0}$ and $a<b$. If $I$ is a time interval, then the following grammar defines the STL language.
\begin{equation}~\label{eqn:stl-gen}
\spec ~ := ~ \true \; | \; f(\outsig(t))\sim 0 \; | \; \neg \spec \; | \;
\spec_1 \wedge \spec_2 \; | \; \spec_1 \U_I \spec_2:~~\sim \in \{ <,\leq,>,\geq,=,\neq \}
\end{equation}
The $\F$ operator is defined as $\F_I \spec \triangleq \true \U_I \spec$, and the $\G$ operator is defined as $\G_I \spec \triangleq \neg (\F _I \neg \spec)$. When omitted, the interval $I$ is taken  to be $[0,\infty)$. The semantics are described informally as follows. The signal $\outsig$ satisfies $f(\outsig)> 0$ at time $t$ if $f(\outsig(t))>0$. It satisfies $\spec = \G_{(0,1]}(f(\outsig)=0)$ if for all time $0< t \leq 1$, $f(\outsig(t))=0$. The signal satisfies $\spec= \F_{[1,2)}(f(\outsig)<0)$ iff there exists a time $t$ such that $1\leq t < 2$ and $f(\outsig(t))<0$. The two-dimensional signal $\outsig=(\outsig_1,\outsig_2)$ satisfies the formula $\spec=(\outsig_1>0)\U_{[2.8,4.5]}(\outsig_2<0)$ iff there is some time $t$ where $2.8 \leq t \leq 4.5$, $\outsig_2(t)<0$, and $\forall t'$ in $[2.8,t)$, $\outsig_1(t')>0$. 

Given a signal $\outsig$ and an STL formula $\spec$, we use computationally efficient methods to determine \emph{how well} $\outsig$ satisfies $\spec$.
The method uses the quantitative semantics for STL, which 
is defined formally in \cite{DonzeM10}, and which we describe informally as follows. The
quantitative semantics defines a function $\rho$ such that a positive sign of
$\rho(\spec,\outsig,t)$ indicates that $(\outsig,t)$ satisfies
$\spec$, and its absolute value estimates the \emph{robustness} of
this satisfaction. If $\phi$ is an inequality of the form
$f(\outsig)>b$, then its robustness is $\rho(\spec,\outsig,t) = f(\outsig(t))-b$.  
When $t$ is omitted, we assume $t=0$ (i.e., $\rho(\spec,\outsig)=\rho(\spec,\outsig,0)$ ).
For the conjunction of two
formulas $\spec := \spec_1 \wedge \spec_2$, we have
$\rho(\spec,\outsig)=\min \left( \rho(\spec_1,\outsig),\rho(\spec_2,\outsig)\right)$,
while for the disjunction $\spec := \spec_1 \vee \spec_2$, we have
$\rho(\spec,\outsig)=\max\left(\rho(\spec_1,\outsig),\rho(\spec_2,\outsig)\right)$.
For a formula with until operator as $\spec := \spec_1 \U_I \spec_2$,
the robustness is computed as $\rho(\spec,\outsig) = \max_{t^\prime\in
  I}\left(\min\left(\rho(\spec_2,\outsig,t^\prime),\min_{t^{\prime\prime}\in
  [t,t^\prime]}\left(\rho(\spec_1,\outsig,t^{\prime\prime})\right)\right)\right).$


\paragraph{Timed Automata}
Let us now use a timed automaton $\mathcal{A}$ that describes the temporal constraints that the input signals should satisfy. The STL formalism can be used as well, but the notion was first developed for timed automata; hence, for simplicity of explanation we assume that we are given a timed automaton that is equivalent to the precondition on the input signals in the STL specification of interest. 

{\bf Recall TA.}

We want to generate a good sample of $N$ timed words of length $n$, each of which is of the form $\gamma = (\delta_1, t_1), \ldots, (\delta_n, t_m)$ where $\delta_i$ are labels of discrete transitions. Our previous work \cite{maxent,BBBK16} propose a sampling method based on a maximal entropy measure which is given by a particular stochastic process \cite{maxent} for timed words of infinite length, and which is the uniform distribution for timed words of finite length \cite{BBBK16}. This \emph{uniform} distribution allows estimating the probability of sampling an incorrect behaviour. Indeed, this distribution assigns the same density of probability $\omega(\vec t) = 1/\Vol(\tpol)$ to every timed vector $\vec t\in\tpol$, where $\Vol(\tpol)=\int_{\tpol} 1 d\vec t$ is the $n$-dimensional volume of $\tpol$. In other words, a sampled timed vector falls in a given subset $A$ of a timed polytope $\tpol$ with probability $\Vol(A)/\Vol(\tpol)$. 
%$\displaystyle \frac{\Vol(A)}{\Vol(\tpol)}$.

The joint law of $\vec T=(T_1,\ldots,T_n)$ is uniquely characterised by its $n$-dimensional CDF which is defined by 
$F(\vec T)=\prob(\vec T\leq \vec t)$ where the partial order $\leq$ is defined by  
$(t_1,\ldots,t_n)\leq (T_1,\ldots,T_n)$ iff $T_i\leq t_i$ for every $i=1 \ldots n$. 
This CDF is usually given by the sequence of conditional CDFs: 
$F_i(t_i\mid t_1,\ldots,t_{i-1})=\prob(T_i\geq t_i\mid T_1= t_1,\ldots, T_{i-1}= t_{i-1})$, 
the following chain rule gives the link between conditional CDF and the CDF of $\vec T$:
$$F(t_1,\ldots,t_n)=F_1(t_1)F_2(t_2\mid t_1)\ldots F_n(t_n\mid t_1,\ldots, t_{n-1}).$$ 
%When the language where $T$ takes its value is recognised by a deterministic timed automaton, a sequence of timed delays $\vec t=t_1,\ldots, t_i$ leads to a unique state $s_{\vec t}$. 
In \cite{BBBK16}, the conditional CDFs $F_i(t_i\mid \vec t)$, used to sample $t_i$, depends only on the current state $(q_{i-1},\x_{i-1})$, that is, $F_i(t_i\mid \vec t)=G_i(t_i \mid (q_{i-1},\x_{i-1}))$ for some conditional CDF $G_i$. 

The conditional CDFs for the uniform distribution on a timed polytope plays a particular role in our subsequent development, and we denote it by $\CDFunif=(\CDFunif_1,\ldots, \CDFunif_n)$. These CDFs are characterised in \cite{BBBK16}, via the definition of conditional PDFs which are the derivatives of the CDFs. The following theorem summarises the results we need.

\begin{theorem}[\cite{BBBK16}]\label{theo:unifCDF}
Given a path in a timed automaton one can compute the CDF $\CDFunif_i$ in polynomial time wrt.~the length of the path. 
These CDFs can be written in the following form $ \CDFunif_i(t_i\mid t_1,\ldots,t_{i-1})=p_i(t_1, \ldots,t_{i-1})/q_i(t_1,\ldots, t_i)$%Le displaystyle est encore en dessous. Attention au guerre d'edition... 
%$\displaystyle \CDFunif_i(t_i\mid t_1,\ldots,t_{i-1})=\frac{p_i(t_1, \ldots,t_{i-1})}{q_i(t_1,\ldots, t_i)}$ 
with $p_i$ and $q_i$ polynomials of degree at most $i$.
\end{theorem}

The uniform sampling should not be confused with the sampling, called \emph{isotropic} in \cite{BBBK16}, that at each step makes a uniform choice amongst the possible delays which is used as a ``default'' sampling in several work (see~\cite{smtcaveat} and references therein). 



\paragraph{Coverage by measuring uniformity degree using the star-discrepancy of a sample of timed vectors}\label{sec:KS}\label{sec:backward}
One way to characterise the uniformity degree of a sample of timed vectors is to use the Kolmogorov-Smirnov test, which is a statistical test to measure how well a sample $S$ of points fits a distribution given by a known CDF $F$. We point out in this section the link between this test and the star-discrepancy. This link allows us to exploit the backward use of $\CDFunif^{-1}$, from the timed polytope to the unit box. %We exploit the ideas from \cite{rosenblatt1952remarks} that were originally developed for the Kolmogorov-Smirnov test of real valued random variables. 

We first recall that the Kolmogorov-Smirnov statistics is defined by the following value (which is a random variable when the sample is drawn at random):
$$\KS(F,S)=\sup_{\vec p\in\R^n}|F(\vec p)-\tilde F_S(\vec p)|$$
where $\tilde F_S$ is the empirical distribution associated with the sample $S$ defined by the CDF 
$\tilde F_S(\vec p)=|\{\vec p'\in S\mid  \vec p'\leq \vec p\}|/|S|,$ which is the ratio of number of points in $S$ that falls in the box $[-\infty, p_1]\times \ldots\times [-\infty, p_n].$ When $F_U$ is the CDF associated to $n$ i.i.d. uniform random variables on $[0,1]$ then  $F_U(\vec p)$ is the volume of the box $[\vec 0, \vec p]$,
and the KS statistics $\KS(F_U,S)$ becomes 
%$$\KS(F_U,S)=\sup_{\vec p\in\R^n}\left|\prod_{i=1..n} p_i- \tilde F_S(\vec p)\right|=D_{\star}(S),$$ which is 
nothing else than $D_{\star}(S)$ the \emph{star-discrepancy} of $S$. This connection is known, see e.g.~\cite{liang2001testing} and reference therein. 
One can translate the multi-dimensional KS statistics  for a sample $S$ (that takes values, in our setting, in a timed polytope) with respect to a CDF $F$  into the KS statistics for the sample $F(S)=\{F(\vec p)\mid \vec p \in P\}$ with respect to the uniform distribution on the unit box. The latter is, as said before, the star-discrepancy of this transformed sample $F(S)$.

In our setting we specify $\CDFunif$ as the CDF of the uniform distribution on a timed polytope. Then,
$$\KS(\CDFunif,S)=\stardisc{\CDFunif(S)}.$$

Note that when $S$ is obtained via uniform (resp.~low-discrepancy) sampling then 
$S=\CDFunif^{-1}(S')$ where $S'$ is a sample of uniform random vectors (resp.~a low-discrepancy sample).
So in that case $\KS(\CDFunif,S)=\stardisc{\CDFunif(\CDFunif^{-1}(S'))}=\stardisc{S'}$. % and the KS test (that requires the KS statistics to be below a threshold) will pass with high probability (resp.~for sure). 


\paragraph{Property Falsification}	

Property falsification is a means of performing automatic bug-finding in system designs.
Given a system model such as (\ref{eq:behaviorfunc}) and a system property $\spec$ provided in the form of an STL formula, 
falsification is a process for finding a parameter value $\param \in \paramset$
such that $y=\behaviorfunc(\param)$ does not satisfy $\spec$, which is denoted $\outsig
\not\models \spec$. Such a behavior $y$ is called a counterexample. 
Note that a counterexample is identified when 
$\rho(\spec,\param)<0$. We call the task of finding a counterexample 
a {\em falsification problem}. 

\paragraph{Optimization and Solvers}	

We formulate the property falsification task as an optimization problem as follows.
\begin{eqnarray} \label{eq:optim1}
\min_{\param \in \paramset} && \rho(\spec,y) \\ \nonumber
s.t. && y=\behaviorfunc(\param)
\end{eqnarray}
This optimization problem is challenging for a number of reasons. First, this optimization problem is mixed in the sense that it contains both discrete and continuous variables. Also, note that the above constraints defined by $\behaviorfunc$ 
are not specified explicitly; rather, the constraint enforces that $y$ is the output signal of model $\transfunc$, given parameters $\param$.
As $\transfunc$ can be a nonlinear hybrid system, for any given $\param$, $y$ can only be determined approximately using numerical simulation. 
Lastly, the cost function $\rho$ is complex and contains discontinuities.
This gives rise to a hard problem of determining the gradients of the cost function, which are often required by traditional continuous optimization techniques. 
For such problems, in general there are no algorithms that can guarantee to find a global optimum \cite{FloudasPardalos2009}, and so we rely on a best effort global optimization techniques. 
%In case the dynamics are continuous, well-known methods for global optimization are only efficient if the cost functions are convex or have some structural properties. Similarly, existing discrete optimization techniques, often faced with the combinatorial explosion issue, are designed to efficiently address specific classes of problems. 

The cost function in (\ref{eq:optim1}) is not convex and not continuous, and so we do not expect to obtain an optimal answer using existing algorithms. We attempt to solve this problem using an approach, called metaheuristics \cite{dreo:hal-01341683}, which attempt to combine the strengths of existing algorithms for discrete and continuous domains, such as Simulated Annealing \cite{Kirkpatrick83optimizationby} and CMA-ES \cite{hansen2006eda}. Also, we note that for most problems, we do not need to identify a true optimum; we merely seek to identify an iterative algorithm that can reduce the cost in (\ref{eq:optim1}) so that $\rho(\spec,y)<0$, which corresponds to a counterexample.

\paragraph{Mapping timed words to real-valued input signals}\label{sec:KS}\label{sec:backward}
We now know how to generate a good sample of $N$ timed words of length $n$, each of which is of the form $\gamma = (\delta_1, t_1), \ldots, (\delta_n, t_m)$ where $\delta_i$ are labels of discrete transitions. Each label $\delta_i$ corresponds to a range of the real-valued signal values $\param$, or more generally to a constraint $g_i(\param, t) \le 0$ for $t \in [t_i, t_{i+1})$ and $\param \in \paramset$. For piecewise constant signals, these constraints are simply interval constraints. For piecewise linear signals, $g_i$ are linear constraints on $\param$ and $t$. 

Hence, given a timed word $\gamma = (\delta_1, t_1), \ldots, (\delta_n, t_n)$, a real-valued input signal corresponding to $\gamma$ satisfies the following constraints, denoted by $C_{\gamma}(\inputs)$:
$$\forall i \in \{1, \ldots, m \}: \inputs (t)= \param, t \in [t_i, t_{i+1}), g_i(\param,t)  \le 0.$$
And we denote this by $\inputs \models C_{\gamma}(\inputs)$.

The optimization problem becomes parameterized with $\gamma$:
\begin{eqnarray} \label{eq:optim2}
\min \rho(\spec, \inputs) \\ \nonumber
s.t. ~\outsig=\behaviorfunc(\inputs) \\ \nonumber
\inputs \models C_{\gamma}(\inputs), \inputs \in \inputset \nonumber
\end{eqnarray}
Let us denote the above optimization problem by $\mathcal{O}_{\gamma}$.

Using the method in \cite{Cosmos}, we generate a set $\Gamma$ of timed words (with good coverage properties). A first {\em abstract} algorithm of our falsification approach is described as follows.
\begin{algorithm}
\caption{Falsification}
\begin{algorithmic}
%\Require  
%\Ensure  		
	        \ForAll{$\gamma \in \Gamma$} 
		\State $\tilde{\rho} = Solve(\mathcal{O}_{\gamma})$
		\If{$\tilde{\rho} \le 0$}
		  \State Exit	
		\EndIf
		\EndFor
		\State No falsifying behavior found. Report coverage of $\spec$
\end{algorithmic}
\end{algorithm}
The function $Solve(\mathcal{O}_{\gamma})$ implements a number of black-box optimization techniques using metaheuristics (such as Simulated Annealing, Evolution Strategies, {\it etc.}), as we have done in \cite{}, to return a best result $\tilde{\rho}$. 

To improve search efficiency, that is to quickly find a falsifying input signal, the values $\tilde{\rho}$ of the objective function can be exploited. Note that if an order on $\Gamma$ can be defined, the search can follow a gridding structure as done in \cite{Valko2018} (Valko)... {\color{red} [to elaborate]}


\paragraph{Coverage as Exploration Performance Measure}	

In addition to the cost valuations, based on $\rho$ in (\ref{eq:optim1}),
the metaheuristic procedure we describe in the sequel utilizes notions of coverage of the parameter space to guide the search.
To capture the amount of coverage that we achieve, we use a metric called {\em cell occupancy}. 

Note that our set of signals corresponds to a set $\inputparamset$ of parameter vectors defining input signals.
Let $\omega=\{ \omega_i | i=1,\ldots,
\numpartition \}$ be a partition of $\inputparamset$. For now, we assume that each partition element,
which we call a \emph{cell}, is rectangular, with each side of equal
length, $\Delta$, called \emph{grid cell size}. A
vector that indicates how many points are in each cell is called a
\emph{distribution}, $\distribution=(n_1,\ldots,n_\numpartition)$,
where each $n_i$ indicates how many points are located in cell $i$.
Cell occupancy is based on the relative number of cells occupied by
points, compared to the total number of cells. Consider the total
number of occupied cells, that is, the number of cells that contain
at least one point, i.e., $\occupiedcellcount =  \sum_{i=1}^{\numpartition} g_i$  
where $g_i = 1$ if  $n_i\geq 1$, and $g_i = 0$ otherwise. Then, the proposed cell occupancy measure is given as
\begin{eqnarray*}
\celloccupancy(\distribution) & = & \frac{\log \occupiedcellcount}{\log \numpartition}.
\end{eqnarray*}
Logarithm functions are used due to the fact that the total number of cells could be very large as compared to the number of occupied cells. The logarithms provide two key features for the cell occupancy measure: (1) they maintain the monotonicity of the measure, and (2) they result in reasonable measure values even for cases where the dimension $\cpdim$ is  large. 


The paper is organized as follows. 
In Sec. \ref{Solvers} we provide an overview of existing metaheuristic methods that we utilize to implement our approach. 
In Section \ref{sec:combination}, we describe our approach, which for iteratively selecting from a collection of different existing optimization solvers.
Section \ref{sec:init} describes how we use information about the progress of the search to initialize the solvers at the start of each iteration. 
In Section \ref{sec:expres} we demonstrate the efficacy of our approach on several challenging examples, including an automotive transmission model, a diesel engine control model, and a model of a hydrogen fuel cell air control system \textcolor{red}{(?)}. Before concluding, we position our approach in relation with existing work using the idea of combining heuristics. 

%%% Local Variables:
%%% mode: latex
%%% TeX-master: "main"
%%% End:


\section{Heuristic Rules for Sequential Solver Execution}



\begin{algorithm}
\caption{Abstract Algorithm for Sequential Solver Execution \label{algoSolverCombination}}
\begin{algorithmic}
%\Require  
%\Ensure  
\State \Comment{{\sf $\solver$ is the solver index}}
\State \Comment{{\sf $\explostateSet[ \solver ] $ is set of exploration states for solver $\solver$}}
\State \Comment{{\sf  $r_{max}$ is the maximal number of rounds}}

\State $r = 1$
\State $blocking = false$
\State $\forall \solver  \in  \solverset, \explostateSet [ \solver ] = \emptyset$   	
\While{$r \le r_{max}$} 
   \If{($blocking$)}
	 \State $\solver = PseudoRand$
	 \State \Comment{{\sf run the pseudo-random solver for $\exectime_{\solver}$ time}}   
	 \State $\{ \bestobj, \explostateSet [\solver] = Run(\solver, \exectime_{\solver})$ 
	  \Else
      %\State \Comment{{\sf  $\solverset$ is the set of all available exploitation-driven solvers}}
       %\State $\overline{\solverset}= \solverset$    \Comment{{\sf  $\overline{\solverset}$ is the set of solvers to run in this round}}
      % \State $\explostateSet_o = \explostateSet$  \Comment{{\sf Store the previous set of exploration states in $\paramset_o$}}
	 \ForAll{$\solver \in \solverset$}  		
	        %\State $\solver = Select(\overline{\solverset})$
 		%\State $\overline{\solverset} = \overline{\solverset} \setminus \solver$
		%\State
		\State $\Gamma = Init(\explostateSet)$
		\State \Comment{{\sf run solver $\solver$ for $\exectime_{\solver}$ time from initial points $\Gamma$}}
  		\State $\{ \bestobj, \explostateSet[\solver] \} = Run(\solver, \param, \exectime_{\solver})$
		\State $\coverage = updateCoverage(\coverage,  \explostateSet)$
      	
 	 \EndFor
         \EndIf	
         \State
\State \Comment{{\sf blocking detection based on coverage and robustness evolution}}
\State $blocking =  DetectBlocking(\coverage, \bestobj)$ 
\State $r++$
\EndWhile
\end{algorithmic}
\end{algorithm}

\subsubsection*{Exploitation-driven versus exploration-driven solvers}
Our abstract algorithm for sequential solver execution is organized in rounds, as shown in Algorithm \ref{algoSolverCombination}. Borrowing the terminology from \cite{}, we divide the solvers used in this work into two rough categories: 
\begin{itemize}
\item {\em Exploitation-driven} : The solvers of this category try to make greedy changes (often small) around the current point. Among the existing solvers of this category, we make use of a number of well-known solvers, namely Simulated Annealing \cite{}, Global Nelder-Mead \cite{}, CMA-ES \cite{}. The solvers of this category are used to explore locally around some potential points. They are efficient when starting at the initial points that are chosen appropriately. They can be used as long as they make the exploration progress, that is the objective values keep getting improved.
\item {\em Exploration-driven}: The solvers of this category make random changes, which could be both large and small, and thus quickly enlarge the exploration space. Such solvers are particularly useful to help the exploration escape local optima where the objective value does not get improved. The pseudo-random search method used in this work belongs to this category. Note also that the pseudo-random search method does not need an initial point, as shown in Algorithm \ref{algoSolverCombination}. 
\end{itemize}

Note that a solver is often configurable, that is its internal parameters can be modified. As an example of internal parameters, if the search solver is Simulated Annealing, the internal parameters include the initial temperature, the number of iterations on one temperature stage and the temperature cooling rate. Tuning a solver configuration is important to achieve its good performance. Indeed, a number of approaches to learning these internal parameters have been proposed in the literature \cite{}. For simplicity of presentation, we omit the internal configurations of the solvers. However, as we will show later, a similar idea will be used to select promising initial points for the exploration. 

Let us now discuss how the exploitation-driven solvers are used. We denote the set of exploitation-driven solvers by $\solverset$. An exploitation-driven solver with index $\solver$ starts from a set of initial points $\Gamma$ (in the parameter space $\paramset$ defining the parameterized input signals) for some execution time $\exectime_{\solver}$. The best objective value obtained after executing a solver is denoted by $\bestobj$. It is also possible to save some or all of the parameter points that the solver has explored in this run, together with their objective function values. By {\em exploration state}, we mean the pair $(\param, \bestobj)$ where $\param$ is a parameter point and $\bestobj$ is its associated objective values. Let $\explostateSet^{\solver}$ denote a set of {\em intermediate exploration states} (the term 'intermediate' here does not refer to their temporal order) for each solver. These points can be used to derive good initializations for the solvers to be executed, which is summarized in the function $Init(\explostateSet)$ in Algorithm \ref{algoSolverCombination}. On the other hand, one need not start uniquely from the best points that have been found so far, the previous explorartion states can indicate promising regions to the next solvers. For example, if the next solver is CMA-ES (the principle of which is to update the mean and the covariance matrix of normally distributed samples in each of its internal iteration), we can define an initial mean and a covariance matrix using only the previously explored points with good objective values. We defer a discussion on this initialization procedure for each solver in Section \ref{sec:init}. 

Note that we store exploration states by different solvers separately, because we want to avoid applying a solver to a point explored previously by the same solver, unless it is one of the best points explored by that solver. Indeed, the random nature of some solvers are only theoretical, therefore under the same configuration and from the same initial point, such solvers in practice often follow the same path. When a solver gets stuck around some local optimum, it is important to be able to detect such blocking situations, and switch to another solver, since different solvers use different search methods and may take the current exploration out of the local optima. 

Let us proceed with a situation where no exploitation-driven solver can make the exploration escape a local optimum. The detection of such situations can be done using coverage and robustness monitoring, as discussed in the subsequent paragraph. To get  the exploration out of a blocking situation, an exploration-driven solver, such as a pseudo-random search method, can be used.  

 

\subsubsection*{Coverage and robustness monitoring for detection of blocking situations}
The exploration is said {\em blocking}, if the exploration, continued with any exploitation driven solver in $\solverset$, does not improve the objective value within some execution time limit. In our falsification context, objective functions are robustness functions of output trajectories of the dynamical system in \ref{eq:behaviorfunc}. A blocking situation can be detected based on the coverage and robustness evolution. More concretely, when both the robustness and coverage values remain stagnant, that is they do not decrease and increase respectively by some predefined amounts, for a number of rounds, we consider that the exploration gets in a blocking situation. Due to the monotonicity of the coverage and robustness evolution with respect to the number of runs, the detection of a blocking state is done by comparing the coverage and the robustness values of the current round and those of the previous round. 


\section{Solver initialization}\label{sec:init}
To find appropriate initial points for the solvers, we inspire from the algorithm configuration tuning methods.  These methods essentially determine the best configurations of an algorithm (designed to solve a given problem) based on the results of running the algorithm on a set of problem instances. The configurations with best performance (with respect to some criteria) will then be used for new problem instances (that arise in the future), since an underlying assumption this approach relies on is that the set of considered instances represent sufficiently well the set of all the possible instances.  

In our falsification setting, we will use these concepts for slightly different meanings. Indeed, in our setting, we have only one problem (defined by a dynamical system and a property), but a number of available solvers; however we can still follow the spirit of the algorithm configuration tuning approach, by letting solver initialization play the role of algorithm configuration.

The solver initialization problem can be formally defined by the following elements
\begin{itemize}
\item $\Gamma$ is the set of initial parameter values.
\item $\solverset$ is the set of solver indices.
%\item $\pi_I$ is a probability measure over the set $I$.
\item $\exectime : \solverset \to \real_+$ is a function associating to every solver of index $\solver$ an execution time.
\item $c(\param, \solver, t)$ is a (random) variable representing the objective function value obtained by running the solver $\solver$ from the initial point $\param \in \paramset$ for $t$ time. 
%\item $C \subset \real$ is the set of possible cost values for all configurations $\theta \in \Theta$ and $\iota \in I$.
%\item $\pi_C$ is a probability measure over the set $C$. Thus $\pi_C(c ~|~ \theta, \iota)$ is the probability that $c$ is the cost of running configuration $\theta$ on instance $\iota$.
%\item $C(\theta) = C(\theta ~|~ \Theta, I, \pi_I, \pi_C, t)$ is the criterion that needs to be optimized with respect to $\theta$. In the most general case it measures in some sense the desirability of the configuration $\theta$.
%\item $T$ is the total amount of time available for experimenting with the given candidate configurations on the available instances before delivering the selected configuration.
\end{itemize}

For a given STL specification $\spec$, let $c(\param, \solver, \exectime_{\solver})$ be the minimal robustness value (over the output traces and with respect to the property $\spec$) obtained by executing the solver $\solver$ under the parameter $\param$ for $\exectime_{\solver}$ generates. We specify a maximum computation time budget for the search from a parameter value because arriving at a global optimum may be infeasible for all possible parameter values. Note that $c(\param, \solver, \exectime_{\solver})$ can be a random variable because of the randomized elements in some solvers. 
%\begin{equation}
%min \{  \rho(\spec, \behaviorfunc(\param)) ~|~ \param \in \paramset \} 
%\end{equation}
%where $\rho(\spec, \behaviorfunc(\param))$ denotes the robustness of the output trajectory $\behaviorfunc(\param)$ with respect to the property $\spec$. The output signal $\behaviorfunc(\param)$ is obtained under the parameter $\param$. 

Our solver initialization problem can thus be thought of as searching in the space $\paramset$ of all possible parameter values for the ones that minimize the objective function $c$. 

%Let $\mathcal{A}$ be the set of search algorithms (such as Simulated Annealing, CMAES, etc.) that are available to us. Mapping to the terminology of the algorithm tuning context, 
% \begin{itemize}
% \item a search algorithm $A \in \mathcal{A}$ (together with one of its internal settings) is an instance in the algorithm tuning terminology. As an example of internal settings of a search algorithm, if the search method is Simulated Annealing, a setting is defined by the initial temperature, the number of iterations on one temperature stage and the temperature cooling rate.  
%\item a {\em configuration} (in the algorithm tuning terminology) is a parameter value $\theta \in \Theta$ at which a search algorithm starts. 
%\item If the specifications of interest are expressed by STL \cite{STL} formulas, $c(\theta, A, t)$ is the minimal robustness value over the simulation traces that the configuration (parameter) $\theta$ generates after running for $t$ time using the algorithm instances (algorithm and one of its setting).
%\end{itemize}
%The measures $\pi_I$ and $\pi_C$ in general are not known, the expected cost is estimated in a Monte Carlo fashion by running the falsification algorithm under a configuration on a training set of instances.

As mentioned earlier, some points that are explored by one algorithm can be stored to be explored later by other algorithms. Note that we may run the same algorithm again but only from the best points. Running a different solver from the explored points results in a new set of explored parameter points together with their cost values, and the set of stored exploration states can quickly become large. It is thus of interest to keep only promising parameter points. To this end, we will make use of a well-known method for tuning algorithm configurations, called F-Race \cite{FRace2010}, which was inspired from racing algorithms in machine learning. The essential idea of these racing algorithms is to evaluate a given set of algorithm configurations iteratively on a sequence of instances. When there is sufficient statistical indication that a configuration performs poorly, it will be excluded from the future search process. To do so, the F-Race method employs Friedman test \cite{Conover1999}. 
In the follwoing we show how this idea can be applied to the problem of solver initialization.


\subsection{F-Race based algorithm for solver initialization}

First, we describe how the F-Race is applied to our solver initialization problem. Each iteration of the procedure corresponds to a solver run. In the first iteration, we sample a set of parameter values $\paramset^0$ which serve as initial candidates. Then, in each iteration, to each candidate, we apply one available solver for some time and record the corresponding cost. The same candidate parameter value can be explored with different search algorithms. The statistical information from the recorded costs is used to decide if a parameter value is not promising at all and thus is dropped. 

Suppose now that the current run is $k$ and let $\Gamma^k$ be the set of the candidate parameter values that are still in the race. Let $m_k = | \Gamma^k | $ be the size of the set $\paramset^k$. Let $m_s$ be the number of solvers, that is $m_s = | \solverset |$. The Friedman test assumes that the costs are $k$ mutually independent $m$-variate random variables. We construct a cost matrix $C^k$ of size $m_s \times m_k$ where the $\solver^{th}$ line is
\begin{eqnarray}\label{eq:C}
c_{\solver}(\param^{q_1})  \; c_{\solver}(\param^{q_2}) \; \ldots  \; c_{\solver}(\param^{q_{m_k})} 
\end{eqnarray}
Element $c_{\solver}(\param^{q_i})$ corresponds to the cost obtained on the surviving parameter value $\param^{q_i}$ by the solver $\solver$. If a parameter value has not been used with a solver $\solver$, it is not included in the matrix. The costs $c_{\solver}(\param^{q_i})$ are ranked in non-decreasing order, that is $q_i \le q_{i'}$ if $c_{\solver}(\param^{q_i}) \le c_{\solver}(\param^{q_{i'}})$. For each parameter value $\param^i$, let $R_{\solver i}$ be the rank of $\param^i$ for the solver $\solver$. Let $$R_i =  \sum_{\solver=1}^{m_s} R_{{\solver} i}$$ 
be the sum of ranks for $\param^i$ with $1 \leq i \leq m_k$. 
%(average ranks are used in case of ties)

To perform the Friedman test \cite{FRace2010}, we determine
\begin{eqnarray*}
\tau & = \displaystyle{ \frac{ (m_k-1) \sum_{j=1}^{m_k} (R_j - (\frac{m_s(m_k+1)}{2})^2 } {\sum_{i=1}^l \sum_{j=1}^{m_k}  R^2_{ij} -  \frac{m_s m_k (m_k+1)^2}{4} }} \nonumber \\ 
\end{eqnarray*}
If the value of $\tau$ is larger than the $1 - \alpha$ quantile of the distribution $\chi^2$  with $(m_k - 1)$ degrees of freedom, the null hypothesis that all parameter values are equivalent is rejected \cite{Papoulis1991}. 

If at the run $k$ this hypothesis is not rejected, we keep the current set of parameter values. If the null hypothesis is rejected, the candidates with the lowest expected rank are considered the most promising parameter values. We then remove from the current set  the values with differences in cost beyond some given threshold.  
 
 \subsection{Iterated F-Race based algorithm for solver initialization}
The F-Race method can also be iterated as follows. Each iteration corresponds to a round, and in each round a number of candidate parameter values remaining from the previous round are used to bias the sampling of new candidates, in view of sampling around the most promising ones. The iterative F-Race can be summarized by three steps in the $r^{th}$ round: 
\begin{itemize}
\item (1) sample $N^r$ candidates based on a probability model; 
\item (2) evaluate the sampled candidates; 
\item (3) update the probability model for the sampling process in the next round.
\end{itemize}
Let $\param \in \real^n$ be a parameter value of our dynamical system such that each component $\param_i \in [\underline{\param}_i, \overline{\param}_i]$. In the $r^{th}$ round, the sampling distribution of $\param_i$ can be a normal distribution $\mathcal{N}(\overline{\param}^r_i, \sigma_i^r)$, where the mean $\overline{\param}^r_i$ is one of the most promising candidates from the previous iteration, selected using their robustness weights. The standard deviation $\sigma_i^r$ in the $r^{th}$ round can be determined by: 
\begin{equation} \label{eq:sigma}
\sigma_i^r = (\overline{\param}_i - \underline{\param}_i) (\frac{1}{N^r})^{r/n}
\end{equation}
which decreases iteration after iteration. The number $N^r$ of candidates can vary, being large at the beginning and decreases gradually. In the first iteration where no information is available, we can sample candidates (or parameter values) according to a uniform distribution. The $r^{th}$ round of the procedure is summarized in Algorithm~\ref{algoFals}, which contains the initialization and execution of explotation-driven solvers in the $r^{th} $ round and can replace the "for all solvers"-loop in Algorithm \ref{algoSolverCombination}. 

\begin{algorithm}
\caption{Solver Initialization and Execution - the $r^{th} $ round \label{algoFals}}
\begin{algorithmic}
%\Require  
%\Ensure  
%\State $k = 1$
%\State $\Gamma^{k-1}=\emptyset$
%\While{$k \le k_{max}$} 
  \State \Comment{{\em Sample new $N^r$ parameter values using distribution $\pi^r$}}
  \State $\Gamma  = \Gamma  \cup  Sample(N^{r}, \pi^r)$
   \State
  \ForAll{$\solver \in \solverset$} 
     \State \Comment{{\em Run solver $\solver$ from parameter values in $\Gamma^k$, if it is not done, for $\exectime_{s}^k$ time}}    
    % \State \Comment{{\em Some intermediate explored points are added in $\Gamma^k$ to produce the new set  $\Gamma^k$} }   
      \State $\{ \bestobj, \explostateSet[\solver] \} = Run(\solver, \Gamma, \exectime_{\solver}^k)$    
      \State  Update the cost table $C^k$ for parameter values and their costs in $\explostateSet[\solver]$ as in (\ref{eq:C})
     %\State $C^k =  C^k \cup c(\Theta^k, A_s, t_{s}^k)$  
     \EndFor
   \State
   \State Run F-Race based algorithm on $C^k$ to exclude the least promising candidates from $\explostateSet$. 
   \State Let $\Gamma$ be the updated parameter value set 
   \State
  \State $r++$  %\Comment{{\em Increment the iteration counter}} 
  \State Update distribution $\pi_i^r$ for each parameter $\param_i$ (using the mean $\overline{\param}^r_i$ and the deviation $\sigma^r_i$ as in (\ref{eq:sigma}))
%\If{}
%\ElsIf{ }
%\EndIf
%\EndWhile
\end{algorithmic}
\end{algorithm}



%\begin{verbatim}
%     fprintf(1,'\n Best Robustness Value of this call = %f', new_obj_best);    
%     fprintf(fileID,'\n Best Robustness Value of this call = %f', new_obj_best);
%     
%     if (new_obj_best<=0)
%        fprintf(fileID,'\n Falsifier Found!');
%        
%        comptime = toc(TotCompTime);
%        fprintf(fileID,'\n Exit! TOTAL Computation time = %f seconds',comptime );
%        error('Falisifier found! Exit normally');
%     end
%     
%     
%     if (call_count==1)  
%         min_robustness=new_obj_best;
%         rob_stagnant = false;
%         rob_improved = true;
%         rob_stagnant_count=0; 
%     else    
%         rob_improved = false;
%         if min_robustness > new_obj_best
%            rob_stagnant = false; 
%            rob_change=(min_robustness - new_obj_best)/min_robustness;
%            if (rob_change > rob_epsilon_percent)
%                rob_improved = true;
%            end   
%            min_robustness=new_obj_best;
%         else 
%             if (~(solver_index==0) && ~(solver_index==4))
%                 rob_stagnant_count=rob_stagnant_count+1; 
%             end
%         end
%         
%         if rob_stagnant_count>rob_stagnant_win
%             rob_stagnant = true;
%         end
%         
%     end 
%     
%     fprintf(1,'\n Best Robustness Value so far = %f', min_robustness);   
%     fprintf(fileID,'\n Best Robustness Value so far = %f', min_robustness);
%     
%     
%    robustness_graph_data=...
%        [robustness_graph_data; [total_nb_sim min_robustness]]; 
%  
%    
%    % the coverage graph is monotonic, we check the evolution of coverage
%    % for non-increase by cov_epsilon
%    % recompute current coverage
%    current_coverage_value = Sys.ComputeLogCellOccupancyCoverage; 
%    % update coverage graph data
%    coverage_graph_data= ...
%       [coverage_graph_data; [total_nb_sim current_coverage_value]]; 
%    
%    solver_index_data=[solver_index_data; solver_index]; 
%   
%   
%    fprintf(1,'\n\n\n\n #Call  SolverID  Robustness  Coverage');
%    fprintf(fileID,'\n\n\n\n #Call  SolverID  Robustness  Coverage');
%    fprintf(1,'\nPseudo-random (0), CMA-ES (1), SA (2), GNM (3)');
%    fprintf(fileID,'\nPseudo-random (0), CMA-ES (1), SA (2), GNM (3)');
%    for iii  = 1:call_count
%      fprintf(1,'\n %d  %d  %12.8f  %12.8f',iii, solver_index_data(iii,1),...
%          robustness_graph_data(iii,2),coverage_graph_data(iii,2));
%      fprintf(fileID,'\n %d  %d  %12.8f  %12.8f',iii, solver_index_data(iii,1),...
%          robustness_graph_data(iii,2),coverage_graph_data(iii,2));     
%    end 
%    
%    
%    l = size(coverage_graph_data,1);
%    
%    if (l>cov_monitoring_length)
%        cov_diff = current_coverage_value - ...
%            coverage_graph_data(l-cov_monitoring_length,2);
%        
%        if (cov_diff<cov_epsilon)
%           stagnant_count = stagnant_count + 1; 
%           %coverage does not increases sufficiently
%        else
%           %coverage increases sufficiently
%           stagnant_count=0;
%        end 
%    
%        if (stagnant_count>cov_monitoring_length) 
%            cov_stagnant=true;
%            fprintf(fileID,'\n Coverage stagnant');
%        else
%            stagnant_count=stagnant_count+1;
%        end
%    end
%    
%    % memorizing the previous optimizer
%    if (~(solver_index==0)) 
%        prev_solver_index = solver_index;
%    end  
%    
%    stagnant_count
%    rob_stagnant
%    cov_monitoring_length
%    local_optimum_stuck=(stagnant_count>=cov_monitoring_length) && rob_stagnant
%    
%    
%    if (~local_optimum_stuck)
%        cov_monitoring_length=cov_monitoring_win;
%        PR_duration=0;
%        solver_index = prev_solver_index + 1;
%        
%%             if (solver_index==3) 
%%                 solver_index=1; %skip GNM
%%             end    
%        if (solver_index>(Nb_Optimizers-1)) 
%            fprintf(1,'\n\n*******\n #%d round(s) of solver calls done', round_count);
%            fprintf(fileID,'\n #%d round(s) of solver calls done', round_count);
%            solver_index = 1;
%            round_count = round_count + 1;
%            
%            rob_stagnant
%            
%            if rob_stagnant
%                %strategy_id = 2 %Thao
%                %solver_index=0
%                strategy_id = 0
%            else
%                if rob_improved
%                    strategy_id = 2
%                else 
%                    strategy_id = 1
%                end    
%            end    
%        end
%        
%    else %if local optima stuck
%        solver_index=0; %use pseudorandom sampling to increase coverage
%        PR_duration=PR_duration+1;
%
%        cov_monitoring_length=PR_duration;
%    end 
%    
%    fprintf(1,'\n Solver call %d done', call_count);
%    fprintf(fileID,'\n Solver call %d done', call_count);
%
%end % end of for-loop call_count
%\end{verbatim}


\bibliographystyle{abbrv}
\bibliography{bibliography} 
\end{document}
