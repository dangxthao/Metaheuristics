\section{Introduction} \label{sec:introduction}

Development of hybrid and cyber-physical systems (CPS) is becoming
increasingly challenging as the designs for modern CPS become more and
more complex.  As these systems are found in many safety-critical
applications, like aircraft, medical devices, and automobiles, it is
vital that they behave in a manner consistent with their
design expectations. Despite this, it is difficult to verify that CPS
designs meet their requirements for complex applications.

% We address this challenge by presenting a new method to automatically
% identify incorrect system behaviors.  Our method searches for faulty
% system behaviors using a metaheuristic falsification approach, which
% leverages a combination of existing search techniques.  The technique
% uses notions of behavior robustness, with respect to requirements, in
% combination with notions of coverage of the search space, to improve
% the overall search performance.

Property \emph{falsification} has garnered much interest recently as a
way to perform automatic bug-finding for complex CPS design
models. Falsification can be thought of as testing where requirements are expressed 
in a formal specification language such as temporal logic.
Two such
languages that are appropriate for CPS applications are metric
temporal logic (MTL) and signal temporal logic (STL)
\cite{Koymans1990,MalerN04} for specifying behaviors defined
using real-valued signals over dense time. A key feature of MTL and
STL is that they are equipped with \emph{robust} semantics, and for a given behavior, methods exist to efficiently compute a
real value, called the robustness which quantifies the requirement satisfaction level of the behavior
\cite{FainekosP06fates,DonzeM10}. A positive robustness value
indicates the behavior satisfies the requirement; a negative
robustness value indicates the behavior does not satisfy the
requirement. Falsification procedures use the robustness as the objective function
for a global optimizer, which seeks a
behavior with negative robustness value. Thus, the optimizer can
be used to automatically find behaviors that violate (falsify) the
requirement. Falsification techniques have been applied to many CPS
systems and are finding application in industry, using tools like
S-TaLiRo and Breach \cite{TaliroLFS11,BreachCAV10}. 

This CPS falsification approach is faced with the following majors challenges.
First, this approach often
requires optimization over continuous-time input signal spaces, whereas
existing optimization solvers expect a finite-dimensional of decision variables. 
An usual approach, such as the ones taken in \cite{BreachCAV10} and 
\cite{Nghiem10}, is to encode input signals spaces of interest using a
finite number of bounded parameters. For example, for a family of
piecewise constant signals with fixed time intervals, the
constant values for the time intervals are parameters treated as the decision variables in the optimization
problem. One drawback of using such fixed parameterizations 
is that the falsification performance depends on the selection of parameterizations, 
which is often based largely on intuition. Another drawback is that, for cases where the
inputs must satisfy non-trivial constraints, encoding these constraints 
into bounded domains for parameters can be difficult. Little attention has
been given to these considerations in the literature, though in 
\cite{DBLP:conf/atva/DeshmukhJKM15} a
falsification strategy featuring variable input discretizations is proposed.

The second challenge is to define meaningful coverage measures to quantify how complete the search is. In the context of CPS such coverage measures should apply to continuous
variables and continuous-time signals. In general coverage measures can be defined with respect to the input space or the behavior space. The latter option is more difficult because the space of all possible behaviors is in general unknown. When an input signal space is parameterized, a coverage measure can be defined on its associated parameter space. Measures like \emph{dispersion} try to capture the size of the empty space between points that have been explored \cite{Esposito04}.
A related and simple measure, partitions the search space into cells and measures the proportion of cells that are occupied by explored points \cite{Skruch2011}. This method is related to the combinatorial entropy notion from the domain of physics to measure the degree of randomness in a distribution of points \cite{Gabbay06}. The \emph{star discrepancy} measure was developed by the statistical community 
to measure the degree to which a set of points are equidistributed \cite{Heinrich03}. Besides, coverage measures can be used to make decisions regarding the search strategy. Coverage measures were used to develop CPS falsification and test generation methods that attempt to maximize the coverage of the search space \cite{DangN09,Dreossi2015,CAV2017}, and these methods were capable of reporting the coverage to the user which is important for the confidence level in the test result, especially when no falsifying behavior is found. The effectiveness of coverage measures depends on the ability of specifying efficiently and accurately the feasible parameter space. When the specification imposes on the input signals complex temporal constraints, the resulting parameter space may be difficult to define. 


%Another core idea is that, given a collection of traditional search
%algorithms, we can judiciously switch between algorithms online, if we
%consider the evolution of both the robustness values and the coverage
%of the search space.


%We introduce a new metaheuristic approach to the search
%problem, which uses measures of both robustness and coverage to make
%decisions about how best to guide the search.
Another crucial factor in the performance of falsification techniques is the efficacy of the global optimization process.
Global optimization algorithms can be broadly classified into one of two
categories: exploration-based and exploitation-based \cite{Blum03}.
Exploration-based methods evaluate points from a widely distributed area 
of the search space, to identify regions that are most promising with 
respect to the cost function. Exploitation-based methods, on the other hand, 
use estimates of the shape of the cost function surface, often locally, to
identify a candidate direction that is most likely to yield a decrease
in cost. Each type of method has strengths and weaknesses. Exploration
methods are not in danger of getting trapped in local minima, but they may 
be inefficient in terms of identifying appropriate directions to search.  By
contrast, exploitation methods are designed to efficiently identify
promising local search directions, but they can get trapped in local optima.
We present a method that synergizes exploration and 
exploitation by adaptively switching between the two strategies.
The tradeoffs between exploitation and exploration have been explored by others, 
for the purposes of falsification for CPS \cite{Ratschan14}. We present a falsification framework that goes further, by incorporating coverage and robustness together to decide when to change search methodology.

In this paper we address the encoding and coverage challenges, by introducing in the falsification framework a new concept of
coverage of the specifications involving temporal constraints on the input signal space. Input signal parameterization can be selected in order to achieve
efficient coverage of the specification. This new concept is built on the approach for uniform
generation of traces of timed automata which are used to specify temporal constraints  \cite{}.
\alex{needs check and citation here}. To improve the optimization efficiency, we propose an iterative procedure that utilizes several
metaheuristics, that is the search methods that can be applied to
different problems, in contrast with heuristics which are often
designed for specific problems. Our procedure can be described as follows. 
In each iteration, our
search procedure considers the evolution of both the best-case
robustness value (the optimization cost) as well as the evolution of a
coverage measure.  We use the information to make online decisions
about when and how to switch from one optimization strategy to
another.  For example, if we are using an exploitation-driven method,
and we decide that the decrease in robustness has ``stalled" (is not
decreasing quickly) and the coverage is ``low", we will switch from
the exploitation-driven method to an alternative exploration-driven
method.  If, alternatively, we are using an exploration-driven method,
and we decide that the coverage is relatively ``high" and the
robustness is near zero, then we will switch to an exploitation-driven
method, using the current best point as an initial condition.
 
Our falsification framework can outperform existing methods with little user intervention or reliance on intuition. We demonstrate this on several 
challenging CPS examples.

The paper is organized as follows. In Section \ref{sec:prlim} we present some preliminaries to provide basic notions for subsequent developments. In Section \ref{sec:specCov} we describe the notion of timed pattern coverage and how it can be exploited in the optimization based falsification problem. Then the paper focuses on the optimization problem. After a brief overview of existing metaheuristic methods that we utilize to solve the optimization problem, in Section \ref{sec:combination}, we describe our approach for systematic combination of different metaheuristics. In Section \ref{sec:expres} we demonstrate the efficacy of our approach on several challenging examples, including an automotive transmission model, a diesel engine control model, and a model of a hydrogen fuel cell air control system \textcolor{red}{(?)}. Before concluding, we position our approach in relation with existing work using the idea of combining heuristics. 

%Section \ref{sec:init} describes how we use information about the progress of the search to initialize the solvers at the start of each iteration. 


%Some of these metaheuristics are
%characterized as exploration-driven methods because of their
%capability of generating diverse candidate solutions. In an opposing
%direction, some are characterized as exploitation-driven methods
%because of their capability of quicky find better solutions in the
%neighborhood of initial candidate solutions.


% We demonstrate the efficacy of our approach on several challenging
% examples, including an automotive transmission model, a diesel
% engine control model, and a model of a hydrogen fuel cell air
% control system.



%%% Local Variables:
%%% mode: latex
%%% TeX-master: "main"
%%% End:
