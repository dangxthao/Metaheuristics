\section{Introduction} \label{sec:introduction}

Development of hybrid and cyber-physical systems (CPS) is becoming
increasingly challenging as the designs for modern CPS become more and
more complex.  As these systems are found in many safety-critical
applications, like aircraft, medical devices, and automobiles, it is
vital that these systems behave in a manner consistent with their
design expectations. Despite this, it is difficult to verify that CPS
designs meet their requirements for complex applications.

% We address this challenge by presenting a new method to automatically
% identify incorrect system behaviors.  Our method searches for faulty
% system behaviors using a metaheuristic falsification approach, which
% leverages a combination of existing search techniques.  The technique
% uses notions of behavior robustness, with respect to requirements, in
% combination with notions of coverage of the search space, to improve
% the overall search performance.

Property \emph{falsification} has garnered much interest recently as a
way to perform automatic bug-finding for complex CPS design
models. Many falsification techniques rely on a requirement
provided in a specification language such as temporal logic.
Two such
languages that are appropriate for CPS applications are metric
temporal logic (MTL) and signal temporal logic (STL)
\cite{Koymans1990,MalerN04}. MTL and STL allow behaviors defined
using real-valued signals over dense time. A key feature of MTL and
STL is that they are equipped with \emph{robust} semantics. This means
that, for a given behavior, methods exist to efficiently compute a
real value, called the robustness, that indicates how well the
behavior does or does not satisfy the requirement
\cite{FainekosP06fates,DonzeM10}. A positive robustness value
indicates the behavior satisfies the requirement; a negative
robustness value indicates the behavior does not satisfy the
requirement. Falsification procedures use the robustness as the cost
for a global optimizer and then rely on the optimizer to search for
behaviors with low robustness value. If the optimizer can identify a
behavior with negative corresponding robustness value, then the
behavior is in violation of the requirement. Thus, the optimizer can
be used to automatically find behaviors that violate (falsify) the
requirement. Falsification techniques have been applied to many CPS
systems and are finding application in industry, using tools like
S-TaLiRo and Breach \cite{TaliroLFS11,BreachCAV10}.

One challenge when using falsification for CPS is that it often
involves optimization over continuous-time input signals, whereas
existing optimization solvers expect a finite set of decision variables. 
The usual approach, such as the one taken in \cite{BreachCAV10} and 
\cite{Nghiem10}, is  to parameterize the input signals space into a 
finite number of bounded parameters. For example, we consider the family of
piecewise constant signals with fixed horizon and a finite number of
constant values, which then become the decision variables for the optimization
problem. One drawback of using fixed parameterizations of the input signal space
to perform falsification
is that a crucial factor in the efficacy
of the technique, the selection of the input parameterization, 
is often based largely on intuition. Another drawback is that, for cases where the
inputs must satisfy non-trivial constraints, encoding these constraints 
into bounded domains for parameters can be difficult. Little attention has
been given to these considerations in the literature, though in 
\cite{DBLP:conf/atva/DeshmukhJKM15} the authors proposed a
falsification strategy featuring variable input discretizations.

Global optimization algorithms can be broadly classified into one of two
categories: exploration-based and exploitation-based \cite{Blum03}.
Exploration-based methods evaluate points from a widely distributed area 
of the search space, to identify regions that are most promising with 
respect to the cost function. Exploitation-based methods, on the other hand, 
use estimates of the shape of the cost function surface, often locally, to
identify a candidate direction that is most likely to yield a decrease
in cost.  Each type of method has strengths and weaknesses. Exploration
methods are not in danger of getting trapped in local minima, but they may 
be inefficient in terms of identifying appropriate directions to search.  By
contrast, exploitation methods are designed to efficiently identify
promising local search directions, but they can get trapped in local optima.
We present a method that synergizes exploration and 
exploitation by adaptively switching between the two strategies.
The tradeoffs between exploitation and exploration have been explored by others, 
for the purposes of falsification for CPS \cite{Ratschan14}. We present a falsification framework that goes further, by incorporating coverage and robustness together to decide when to change search methodology.

Coverage is an important consideration in testing for CPS, though it can be 
difficult to define and measure.
One challenge is in defining meaningful coverage measures that apply to continuous
variables.
Measures like \emph{dispersion} try to capture the size of the empty space between points that have been explored \cite{Esposito04}.
A related and simple measure, partitions the search space into cells and measures the proportion of cells that are occupied by explored points \cite{Skruch2011}.
This method is related to the combinatorial entropy notion from the domain of physics to measure the degree of randomness in a distribution of points \cite{Gabbay06}.
The \emph{star discrepancy} measure was developed by the statistical community 
to measure the degree to which a set of explored points are equidistributed
\cite{Heinrich03}. 
Coverage measures were used to develop CPS falsification and test generation methods 
 that 
attempt to maximize the amount of coverage of the search space \cite{DangN09,Dreossi2015,CAV2017}, and these methods were
capable of reporting coverage back to the user.
Our falsification framework uses coverage measures to make decisions regarding the search strategy.

A novel feature of our falsification framework is that it can 
target coverage based on the specifications as well as the system states.
%We propose a
%metaheuristics framework that adaptively selects falsification search 
%strategies based on coverage and robustness. 
Our approach dynamically adjusts input parameterization in order to achieve
efficient coverage of the input space, even when inputs are subject to
complex temporal constraints. Our approach builds on uniform
distributions of traces of timed automata presented in \cite{}.
\alex{needs check and citation here}.
%Another core idea is that, given a collection of traditional search
%algorithms, we can judiciously switch between algorithms online, if we
%consider the evolution of both the robustness values and the coverage
%of the search space.


%We introduce a new metaheuristic approach to the search
%problem, which uses measures of both robustness and coverage to make
%decisions about how best to guide the search.

Our method is an iterative procedure that utilizes several
metaheuristics, that is the search methods that can be applied to
different problems, in contrast with heuristics which are often
designed for specific problems. Our method can be described as follows. 
In each iteration, our
search procedure considers the evolution of both the best-case
robustness value (the optimization cost) as well as the evolution of a
coverage measure.  We use the information to make online decisions
about when and how to switch from one optimization strategy to
another.  For example, if we are using an exploitation-driven method,
and we decide that the decrease in robustness has ``stalled" (is not
decreasing quickly) and the coverage is ``low", we will switch from
the exploitation-driven method to an alternative exploration-driven
method.  If, alternatively, we are using an exploration-driven method,
and we decide that the coverage is relatively ``high" and the
robustness is near zero, then we will switch to an exploitation-driven
method, using the current best point as an initial condition.
 
Our falsification framework can outperform existing methods with little user intervention or reliance on intuition. We demonstrate this on several 
challenging CPS examples.


%Some of these metaheuristics are
%characterized as exploration-driven methods because of their
%capability of generating diverse candidate solutions. In an opposing
%direction, some are characterized as exploitation-driven methods
%because of their capability of quicky find better solutions in the
%neighborhood of initial candidate solutions.


% We demonstrate the efficacy of our approach on several challenging
% examples, including an automotive transmission model, a diesel
% engine control model, and a model of a hydrogen fuel cell air
% control system.



%%% Local Variables:
%%% mode: latex
%%% TeX-master: "main"
%%% End:
