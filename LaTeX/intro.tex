\section{Introduction} \label{sec:introduction}

Development of hybrid and cyber-physical systems (CPS) is becoming
increasingly challenging as the designs for modern CPS become more and
more complex.  As these systems are found in many safety-critical
applications, like aircraft, medical devices, and automobiles, it is
vital that these systems behave in a manner consistent with their
design expectations. Despite this, it is difficult to verify that CPS
designs meet their requirements for complex applications.

% We address this challenge by presenting a new method to automatically
% identify incorrect system behaviors.  Our method searches for faulty
% system behaviors using a metaheuristic falsification approach, which
% leverages a combination of existing search techniques.  The technique
% uses notions of behavior robustness, with respect to requirements, in
% combination with notions of coverage of the search space, to improve
% the overall search performance.

Property \emph{falsification} has garnered much interest recently as a
way to perform automatic bug-finding for complex CPS design
models. Many existing falsification techniques rely on a requirement
provided in some precise and unambiguous logical language. Two such
languages that are appropriate for CPS applications are metric
temporal logic (MTL) and signal temporal logic (STL)
\cite{Koymans1990,MalerN04}. MTL and STL allow behaviors defined
using real-valued signals over dense time. A key feature of MTL and
STL is that they are equipped with \emph{robust} semantics. This means
that, for a given behavior, methods exist to efficiently compute a
real value, called the robustness, that indicates how well the
behavior does or does not satisfy the requirement
\cite{FainekosP06fates,DonzeM10}. A positive robustness value
indicates the behavior satisfies the requirement; a negative
robustness value indicates the behavior does not satisfy the
requirement. Falsification procedures use the robustness as the cost
for a global optimizer and then rely on the optimizer to search for
behaviors with low robustness value. If the optimizer can identify a
behavior with negative corresponding robustness value, then the
behavior is in violation of the requirement. Thus, the optimizer can
be used to automatically find behaviors that violate (falsify) the
requirement. Falsification techniques have been applied to many CPS
systems and are finding application in industry, using tools like
S-TaLiRo and Breach \cite{TaliroLFS11,BreachCAV10}.

One challenge is in determining how to provide guidance to
the designers when the falsification method fails to identify a
violating behavior.  Because falsification techniques are \emph{best
  effort} methods, they are not guaranteed to find a falsifying case,
if one exists.  In other words, the absence of a falsifying case does
not indicate that the system is incapable of exhibiting bad behaviors.
So the question is, if no falsifying trace is found by the optimizer,
what can be concluded? In these cases, one useful piece of information
that can be reported back to the designers is the degree to which the
search space was covered in the attempt to falsify the property.
Previous work considered falsification methods that attempt to
maximize the amount of coverage of the search space when performing
the falsification \cite{Dreossi2015,CAV2017}, and these methods were
capable of reporting coverage back to the user.

Another challenge when using falsification for CPS is that it often
involves optimization over continuous-time input signals whereas
existing optimization solvers expects finite sets of variables. The usual approach
consists in parameterizing the input signals space into a finite number
of bounded parameters. For example, we consider the family of
piecewise constant signals with fixed horizon and finite number of
constant values, which then become the variables of the optimization
problem.  This approach however leads to the following important issues:
\begin{itemize}
\item The choice of the parameterization is crucial to the outcome of
  the falsification; too many parameters will lead to a richer class
  of signals, but more difficult problem to solve;
\item If falsification fails, the coverage measure obtained
  is limited to the coverage of the parameterized input signal space;
\item When inputs must satisfy non trivial constraints, encoding
  these constraints into bounded domains for parameters can be non
  trivial as well and lead to more challenging optimization problems.
\end{itemize}
Although a problem formulation resulting from input parameterization
can be as important or even more important than actually solving the
problem, few recent works have considered this issue. In
\cite{DBLP:conf/atva/DeshmukhJKM15}, the authors proposed a
falsification strategy featuring variable input discretizations.

Applying falsification thus often involves solving or trying to solve
several successive optimization problems, where the user chooses the
next solving strategy based on how robustness values evolved or how
much and what type of coverage is desired. In this paper, we propose a
metaheuristics framework to automate this process of iterating through
problem formulations and solving strategies. Our approach makes it
possible to adjust input parameterization in order to achieve
efficient coverage of the input space, even when inputs are subject to
complex temporal constraints. Our approach builds on uniform
distributions of traces of timed automata presented in \cite{}.
\alex{needs check and citation here}.
Another core idea is that, given a collection of traditional search
algorithms, we can judiciously switch between algorithms online, if we
consider the evolution of both the robustness values and the coverage
of the search space.

Global optimization algorithms can be broadly classified into one of two
categories: exploration-based and exploitation-based.
Exploration-based methods use various techniques, including stochastic
methods, to investigate a widely distributed area of the search space,
to identify regions that are most promising, with respect to the cost
function. Exploitation-based methods, on the other hand, use
estimates of the shape of the cost function surface, often locally, to
identify a candidate direction that is most likely to yield a decrease
in cost.  Each method has its strengths and weaknesses. Exploration
methods are able to search the decision space broadly and are not in
danger of getting trapped in local minima, but they may be inefficient
in terms of identifying appropriate directions to search.  By
contrast, exploitation methods are designed to efficiently identify
promising local search directions, but they are in danger of getting
trapped in local optima.

%We introduce a new metaheuristic approach to the search
%problem, which uses measures of both robustness and coverage to make
%decisions about how best to guide the search.

Our method is an iterative procedure that utilizes several
metaheuristics, that is the search methods that can be applied to
different problems, in contrast with heuristics which are often
designed for specific problems. Some of these metaheuristics are
characterized as exploration-driven methods because of their
capability of generating diverse candidate solutions. In an opposing
direction, some are characterized as exploitation-driven methods
because of their capability of quicky find better solutions in the
neighborhood of initial candidate solutions.  In each iteration, our
search procedure considers the evolution of both the best-case
robustness value (the optimization cost) as well as the evolution of a
coverage measure.  We use the information to make online decisions
about when and how to switch from one optimization strategy to
another.  For example, if we are using an exploitation-driven method,
and we decide that the decrease in robustness has ``stalled" (is not
decreasing quickly) and the coverage is ``low", we will switch from
the exploitation-driven method to an alternative exploration-driven
method.  If, alternatively, we are using an exploration-driven method,
and we decide that the coverage is relatively ``high" and the
robustness is near zero, then we will switch to an exploitation-driven
method, using the current best point as an initial condition.
 




% We demonstrate the efficacy of our approach on several challenging
% examples, including an automotive transmission model, a diesel
% engine control model, and a model of a hydrogen fuel cell air
% control system.



%%% Local Variables:
%%% mode: latex
%%% TeX-master: "main"
%%% End:
