\section{Preliminaries} \label{sec:prlim}
\paragraph{Signal temporal logic.} 

We assume that the correct or expected behaviors for system (\ref{eq:behaviorfunc}) are provided in an unambiguous form that can be efficiently measured and quantified. For this purpose, we use the signal temporal logic (STL) language to define the system specifications.
STL is a modal logic that is well-defined over discrete or real-valued signals and discrete or continuous time \cite{MalerN04}.
STL is appropriate for specifying correct behavior for CPSs, as it can be defined over the real-valued, continuous-time signals that characterize CPS behaviors. 

We present an overview of STL and refer the reader to \cite{MalerN04} for a detailed presentation.
An STL formula $\spec$ consists of atomic predicates along with logical and temporal connectives.
Atomic predicates are defined over signal values and have the form $ f(\outsig(t))\sim 0$, where $f$ is a scalar-valued function over the signal $y$ evaluated at time $t$, and $\sim \in \{ <,\leq, >, \geq, =, \neq \}$.
Temporal operators ``always'' ($\G$), ``eventually'' ($\F$), and ``until'' ($\U$) have the usual meaning and are scoped using intervals of the form $(a,b)$, $(a,b]$, $[a,b)$, $[a,b]$, or $(a,\infty)$, where 
$a,b\in \real_{\geq 0}$ and $a<b$. If $I$ is a time interval, then the following grammar defines the STL language.
\begin{equation}~\label{eqn:stl-gen}
\spec ~ := ~ \true \; | \; f(\outsig(t))\sim 0 \; | \; \neg \spec \; | \;
\spec_1 \wedge \spec_2 \; | \; \spec_1 \U_I \spec_2:~~\sim \in \{ <,\leq,>,\geq,=,\neq \}
\end{equation}
The $\F$ operator is defined as $\F_I \spec \triangleq \true \U_I \spec$, and the $\G$ operator is defined as $\G_I \spec \triangleq \neg (\F _I \neg \spec)$. When omitted, the interval $I$ is taken  to be $[0,\infty)$. The semantics are described informally as follows. The signal $\outsig$ satisfies $f(\outsig)> 0$ at time $t$ if $f(\outsig(t))>0$. It satisfies $\spec = \G_{(0,1]}(f(\outsig)=0)$ if for all time $0< t \leq 1$, $f(\outsig(t))=0$. The signal satisfies $\spec= \F_{[1,2)}(f(\outsig)<0)$ iff there exists a time $t$ such that $1\leq t < 2$ and $f(\outsig(t))<0$. The two-dimensional signal $\outsig=(\outsig_1,\outsig_2)$ satisfies the formula $\spec=(\outsig_1>0)\U_{[2.8,4.5]}(\outsig_2<0)$ iff there is some time $t$ where $2.8 \leq t \leq 4.5$, $\outsig_2(t)<0$, and $\forall t'$ in $[2.8,t)$, $\outsig_1(t')>0$. 

Given a signal $\outsig$ and an STL formula $\spec$, we use computationally efficient methods to determine \emph{how well} $\outsig$ satisfies $\spec$.
The method uses the quantitative semantics for STL, which 
is defined formally in \cite{DonzeM10}. The
quantitative semantics defines a function $\rho$ such that a positive sign of
$\rho(\spec,\outsig,t)$ indicates that $(\outsig,t)$ satisfies
$\spec$, and its absolute value estimates the \emph{robustness} of
this satisfaction. %If $\phi$ is an inequality of the form
%$f(\outsig)>b$, then its robustness is $\rho(\spec,\outsig,t) = f(\outsig(t))-b$.  
%When $t$ is omitted, we assume $t=0$ (i.e., $\rho(\spec,\outsig)=\rho(\spec,\outsig,0)$ ).
%For the conjunction of two
%formulas $\spec := \spec_1 \wedge \spec_2$, we have
%$\rho(\spec,\outsig)=\min \left( \rho(\spec_1,\outsig),\rho(\spec_2,\outsig)\right)$,
%while for the disjunction $\spec := \spec_1 \vee \spec_2$, we have
%$\rho(\spec,\outsig)=\max\left(\rho(\spec_1,\outsig),\rho(\spec_2,\outsig)\right)$.
%For a formula with until operator as $\spec := \spec_1 \U_I \spec_2$,
%the robustness is computed as $\rho(\spec,\outsig) = \max_{t^\prime\in
%  I}\left(\min\left(\rho(\spec_2,\outsig,t^\prime),\min_{t^{\prime\prime}\in
%  [t,t^\prime]}\left(\rho(\spec_1,\outsig,t^{\prime\prime})\right)\right)\right).$

\paragraph{Input signal parametrization.} We consider dynamical system models, represented by the following mapping:
\begin{eqnarray}
\outsig = \transfunc (\paramsorig, \inputs),
\end{eqnarray}
where $\paramsorig \in \paramsorigset$ is a set of parameters that affect the system behaviors, and $\inputs \in \inputset$ is a function of time that represents the inputs to the system.
Parameters $\paramsorig$ could contain a set of system initial conditions as well as some finite set of variables that affect how the system maps inputs to outputs.
Each $\inputs$ is a function $\timeinterval \mapsto \inputspace$, where $\timeinterval$ is an interval (either continuous or discrete) from $0$ to some finite value, and $\inputspace$ is some metric space of finite dimension.
Similarly, we assume that each output signal $\outsig \in \outsigset$ is a function $\timeinterval \mapsto \outspace$, where $\outspace$ is some metric space of finite dimension.

Input $\inputs$ is generally taken from an infinite-dimensional signal space (i.e., these can be partial functions over a continuous time-domain), but we restrict our investigation to the class of  input signals that are finitely parameterizable. That is, we assume that any input signal $\inputs$ 
can be uniquely characterized by a set of $m$ parameters, whose valuation $\inputparams =(\inputparams_1,\ldots,\inputparams_m) \in \inputparamset$ is in a subset of an $m$-dimensional metric space. For example, a right-continuous piecewise constant input signal $\inputs:\timeinterval \rightarrow \real$, where $\timeinterval=[ 0,T ]$, with discontinuities occurring at monotonically increasing instants $\tau_1,\ldots, \tau_m$, where $0=\tau_1<\tau_m<T$, can be uniquely characterized by $m$ values $\inputs(\tau_i)$, and we can define $\inputparams_i=\inputs(\tau_i)$. As another example, a piecewise linear signal over the same set of $m$ time points used to define the non-differentiable instants can be also uniquely characterized by $m$ values $\inputparams_i = \inputs(\tau_i)$, that is $\forall t \in [\tau_i, \tau_{i+1})$, $\inputs(t) = \inputparams_i  + 
\frac{\inputparams_{i+1} - \inputparams_i}{\tau_{i+1} -  \tau_i}(t-\tau_i)$.

In most existing work, the time instants are fixed, and the signal values at these instants are the decision variables for the search. In the present work, we build on that method by extending the class of input signals in two ways. First, the time instants are also parameterized and thus become decision variables in the search space. Second, we allow temporal constraints on the inputs, the goal of which, as mentioned in the introduction, is to reduce the search space to the input signals of interest. The idea being that, if the specification involves some temporal constraints on the input signals, focusing only on signals that satisfy the constraints will increase the falsification efficiency. 

We call a function that maps input parameters to input functions a {\em parametrization function}. The input parameters are put into two parameter vectors: (1) the vector of time instant parameters $\instantparams = (\tau_1,\ldots, \tau_m)$ taking values in $\instantparamset=\timeinterval^m$, and (2) the vector of value parameters $\valueparams= (\inputparams_1, \ldots, \valueparams_m)$ taking values in $\inputparamset$. We define an augmented set of parameters $\param = (\paramsorig, \instantparams, \valueparams)$, where $p \in \paramset = \paramsorigset \times \instantparamset  \times \inputparamset$. Given a parametrization function $\inputgenerator: \instantparamset \times \valueparamset \rightarrow \inputset$, the dynamical system can be described as follows:
\begin{eqnarray} \label{eq:behaviorfunc}
y &=& \behaviorfunc(\param),
\end{eqnarray}
where $\behaviorfunc(\param) = \transfunc (\paramsorig, \inputgenerator(\tau, \inputparams))$.  Note that $\inputgenerator$ is absorbed into the definition of $\behaviorfunc$.


\section{Timed pattern coverage guided falsification} \label{sec:specCov}
We use timed automata  to describe the temporal constraints that the input signals should satisfy \footnote{The STL formalism can be used as well, but the generation method was developed for timed automata; hence, for simplicity of explanation we assume that we are given a timed automaton that is equivalent to the precondition on the input signals in the STL specification of interest.}, thus intuitively timed pattern coverage measures the portion of all possible timed patterns of the input signals satisfying the specification that have been explored. Timed patterns in the context of timed automata can be defined using timed words. Let us first briefly recall timed automata. A \emph{timed automaton} $\mathcal{A} = (\Loc, \Clocks, \InitLoc, \Alphabet, \Edg, \Inv)$ is a tuple where $\Loc$ is a finite set of locations; $\Clocks$ is a finite set of clocks; $\InitLoc \in \Loc$ is the initial location; $\Edg$ is a finite set of transitions. Each transition is the form $\delta = (q,\guard,a, \reset,q')$ where $q, q' \in \Loc$ are the source and destination locations; $\guard$ is the guard, and $\alpha$ is a label; $\reset$ is the reset map; $\Inv$ associates with each location $q$ a conjunction of clock constraints, called the invariant of $q$. A state of $\mathcal{A}$ is a pair $(q,x)$ where $q \in \Loc$ and $x$ is a clock valuation. The initial state of $\mathcal{A}$ is $(\InitLoc, x_0)$. The transitions of the automaton are of two types: timed transitions and discrete transitions. Timed transitions correspond to the evolution of the clocks within a location as long as the clock valuation satisfies the invariant of the location. Concerning discrete transitions, for $\alpha \in \Alphabet$, if the transition $\edg = (q,\guard,\alpha,\reset,q')$ is enabled at the state $(q, x)$ (that is $x$ satisfies the guard $G$), the discrete transition from $q$ to $q'$ can take place (if the clock valuation after resetting satisfies the invariant of $q'$). A run is an alternating sequence $(\InitLoc,x_0)\xrightarrow{\delta_1,\tau_1}(q_1,x_1)\ldots \xrightarrow{\delta_n,\tau_m}(q_m,x_m)$ of states and timed transitions with the following updating rules: $q_{i}$ is the successors of $q_{i-1}$ by $\delta_i$, the vector $x_{i-1}+\tau$ must satisfy the guard of $\delta_i$ and $x_i$ is updated by applying the reset map to $(x_{i-1}+\tau)$. In our formalism, the labels of discrete transitions correspond to the predicates on the signal values. A timed word of length $m$ is $\gamma = (a_1, \tau_1), \ldots, (a_m, \tau_m)$ where $a_i$ are labels of discrete transitions and $\tau_i$ are the time durations between two consecutive transitions; $(\tau_1, \ldots, \tau_m)$ is called a timed vector and $(a_1, \ldots, a_m)$ discrete pattern. Each label $a_i$ corresponds to a predicate on the real values $\valueparams$ of the signal of the form $\pi_i(\valueparams) \le 0$. 

Timed words together with the predicates associated to the transition labels provide the timed patterns for the valid input signals that is those satisfying the time constraints. The work \cite{BBBK16} proposes a method for uniform random sampling of timed words of finite length. This \emph{uniform} distribution allows estimating the probability of sampling a falsifying behaviour. Hence, the coverage of a given set of timed patterns can be characterised by its ``uniformity degree'', that is how well this set fits the uniform distribution. This uniformity degree notion can be formalized using the Kolmogorov-Smirnov test, which is a statistical test to measure how well a sample $S$ of points fits a given distribution described by a cumulative distribution function (CDF) $F$. More concretely, the Kolmogorov-Smirnov (KS) statistics is defined by the following value (which is a random variable when the sample is drawn at random):
$$\KS(F,S)=\sup_{p\in\R^n}|F(p)-\tilde F_S(p)|$$
where $\tilde F_S$ is the empirical distribution associated with the sample $S$ defined by the CDF 
$\tilde F_S(p)=\displaystyle{\frac{|\{p'\in S\mid  p'\leq p\}|}{|S|}},$ which is the ratio of number of points in $S$ that are in the box $[-\infty, p_1]\times \ldots\times [-\infty, p_n].$ 
%When $F_U$ is the CDF associated to $n$ i.i.d. uniform random variables on $[0,1]$ then  $F_U(\vec p)$ is the volume of the box $[\vec 0, \vec p]$, and the KS statistics $\KS(F_U,S)$ becomes 
%%$$\KS(F_U,S)=\sup_{\vec p\in\R^n}\left|\prod_{i=1..n} p_i- \tilde F_S(\vec p)\right|=D_{\star}(S),$$ which is 
%nothing else than $D_{\star}(S)$ the \emph{star-discrepancy} of $S$. This connection is known, see e.g.~\cite{liang2001testing} and reference therein. 
%Given a set of timed patterns with the same pattern (that is with the same sequence of discrete transitions), to measure its uniformity degree, we apply the above notion to the set of their time vectors $\tau = (\tau_1, \ldots, \tau_m)$ with respect to the timed polytope (corresponding to the set of all reachable timed vectors when the timed automaton follows the sequence of discrete transitions). It is worth noting that one can translate the multi-dimensional KS statistics for a sample $S$ that takes values in a timed polytope $\mathcal{P}_{\tau}$ with respect to a CDF $F$  into the KS statistics for the sample $F(S)=\{F(\tau)\mid  \tau \in \mathcal{P}_{\tau}\}$ (resulting from applying $F$ to $S$) with respect to the uniform distribution on the unit box. The latter turns out to be the star-discrepancy of this transformed sample $F(S)$. The measure can also be extended to sets of timed patterns  with different patterns.

We apply this multi-dimensional KS statistics to timed patterns as follows. We first assume that a set of timed patterns of interest has the same discrete pattern (that is with the same sequence of discrete transitions). We consider the associated set $S$ of timed vectors $\tau = (\tau_1, \ldots, \tau_m)$ which lie in a timed polytope $\mathcal{P}_{\tau}$ (that is the set of all reachable timed vectors when the timed automaton follows this discrete pattern) \cite{BBBK16}. The set $S$ can be thought of as a set of points sampled from the timed polytope $\mathcal{P}_{\tau}$ according to some CDF $F$. Note that the uniform random generation method  \cite{BBBK16}  that we use guarantees that $F$ is indeed the CDF of the uniform distribution on $\mathcal{P}_{\tau}$. It can be proven that the KS statistics of such a set $S$ is equal to the KS statistics of the set $F(S)=\{F(\tau)\mid  \tau \in \mathcal{P}_{\tau}\}$ (resulting from applying $F$ to $S$) sampled uniformly in the unit box \cite{rosenblatt1952}. The latter turns out to be the star-discrepancy of this transformed sample $F(S)$. The star-discrepancy notion has previously been used for coverage measure and can be evaluated \cite{DangN09}. This analysis can also be extended to sets of timed patterns with different discrete patterns.

%We can use this notion of "uniformity degree" to measure the coverage of a set of timed patterns. This is particularly useful when a set of timed patterns does not  falsify a property because it indicates the portion of tested timed patterns with respect to all possible timed patterns.

%In our setting we specify $\CDFunif$ as the CDF of the uniform distribution on a timed polytope. Then,
%$$\KS(\CDFunif,S)=\stardisc{\CDFunif(S)}.$$

%Note that when $S$ is obtained via uniform (resp.~low-discrepancy) sampling then 
%$S=\CDFunif^{-1}(S')$ where $S'$ is a sample of uniform random vectors (resp.~a low-discrepancy sample).
%So in that case $\KS(\CDFunif,S)=\stardisc{\CDFunif(\CDFunif^{-1}(S'))}=\stardisc{S'}$. % and the KS test (that requires the KS statistics to be below a threshold) will pass with high probability (resp.~for sure). 


\paragraph{Property falsification using optimization.}	

Having defined all the necessary notions, we can now formally state our falsification problem.
Given a system model such as (\ref{eq:behaviorfunc}) and a system property $\spec$ provided in the form of an STL formula, 
falsification is a process for finding a parameter value $\param \in \paramset$
such that $y=\behaviorfunc(\param)$ does not satisfy $\spec$, which is denoted $\outsig
\not\models \spec$. Such a behavior $y$ is called a counterexample. 
Note that a counterexample is identified when 
$\rho(\spec,\param)<0$. We call the task of finding a counterexample 
a {\em falsification problem}. 

%\paragraph{Optimization and Solvers}	

We formulate the property falsification task as an optimization problem as follows.
\begin{eqnarray} \label{eq:optim1}
\min_{\param \in \paramset} && \rho(\spec,y) \\ \nonumber
s.t. && y=\behaviorfunc(\param)
\end{eqnarray}
This optimization problem is challenging for a number of reasons. First, this optimization problem is mixed in the sense that it contains both discrete and continuous variables. Also, note that the above constraints defined by $\behaviorfunc$ 
are not specified explicitly; rather, the constraint enforces that $y$ is the output signal of model $\transfunc$, given parameters $\param$.
As $\transfunc$ can be a nonlinear hybrid system, for any given $\param$, $y$ can only be determined approximately using numerical simulation. 
Lastly, the cost function $\rho$ is complex and contains discontinuities.
This gives rise to the hard problem of determining the gradients of the cost function, which are often required by traditional continuous optimization techniques. 
For such problems, in general there are no algorithms that can guarantee to find a global optimum \cite{FloudasPardalos2009}, and so we rely on a best effort global optimization techniques. 
%In case the dynamics are continuous, well-known methods for global optimization are only efficient if the cost functions are convex or have some structural properties. Similarly, existing discrete optimization techniques, often faced with the combinatorial explosion issue, are designed to efficiently address specific classes of problems. 

The cost function in (\ref{eq:optim1}) is not convex and not continuous, and so we do not expect to obtain an optimal answer using existing algorithms. We attempt to solve this problem using an approach, called metaheuristics \cite{dreo:hal-01341683}, which attempt to combine the strengths of existing algorithms for discrete and continuous domains. Examples of metaheuristic methods include Simulated Annealing \cite{Kirkpatrick83optimizationby} and CMA-ES \cite{hansen2006eda}. Also, note that for most falsification problems, we do not need to identify a true optimum; we merely seek to identify an iterative algorithm that can reduce the cost in (\ref{eq:optim1}) so that $\rho(\spec,y)<0$, which corresponds to a counterexample.

%\paragraph{Mapping timed words to real-valued input signals}\label{sec:KS}\label{sec:backward}
%We now know how to generate a good sample of $N$ timed words of length $m$, each of which is of the form $\gamma = (\delta_1, t_1), \ldots, (\delta_m, \tau_m)$ where $\delta_i$ are labels of discrete transitions. Each label $\delta_i$ corresponds to a range of the real-valued signal values $\param$, or more generally to a constraint $g_i(\param, t) \le 0$ for $t \in [\taut_i, \taut_{i+1})$ and $\param \in \paramset$. For piecewise constant signals, these constraints are simply interval constraints. For piecewise linear signals, $g_i$ are linear constraints on $\param$ and $t$. 

\paragraph{Combining optimization and timed patterns generation.} Considering  both time instants and values as optimization variables is expensive not only because of high dimension but also because it is difficult to capture and handle time constraints on the input signals. An intuitive approach is to first generate a candidate solution and then check whether it satisfies the time constraints. The candidate solution is rejected if it is not valid (i.e., if it does not satisfy the temporal constraints), and a new candidate solution is generated. This method is clearly not efficient, especially when the space of valid input signals is small compared to the entire space of input signals. It is thus desirable to generate only valid candidate solutions. To this end, we propose to combine optimization and timed pattern generation: we generate timed patterns from the timed automaton (describing the time constraints) and then for each timed pattern (with now fixed-instant parameters) the optimization problem is solved using only the signal values as decision variables. The approach is described more concretely as follows.

Given a timed word $\gamma = (\delta_1, \tau_1), \ldots, (\delta_m, \tau_m)$, when a real-valued input signal $\inputs$ is generated from $\gamma$ such that $\inputs$ satisfies the following constraints:
$$\forall i \in \{1, \ldots, m \}: \inputs (t)= \inputparams_i, t \in [\tau_i, \tau_{i+1}), \pi_i(\inputparams)  \le 0,$$
then we say $(\instantparams, \inputparams) \models C_{\gamma}(\instantparams, \inputparams)$. The optimization problem is then parameterized by $\gamma$ as follows:
\begin{eqnarray} \label{eq:optim2}
\min \rho(\spec, \inputs) \\ \nonumber
s.t. ~\outsig=\behaviorfunc(\param) \\ \nonumber
(\instantparams, \inputparams) \models C_{\gamma}(\instantparams, \inputparams), \param = (\paramsorig, \instantparams, \inputparams) \in \paramset . \nonumber
\end{eqnarray}
Let us denote the above optimization problem by $\mathcal{O}_{\gamma}$. A first {\em abstract} algorithm of our falsification approach is described as follows.

\begin{algorithm}
\caption{Falsification}
\begin{algorithmic}
%\Require  
%\Ensure  		
		\State $\Gamma = Generate(\mathcal{A},m)$
	        \ForAll{$\gamma \in \Gamma$} 
		\State $\tilde{\rho} = Solve(\mathcal{O}_{\gamma})$
		\If{$\tilde{\rho} \le 0$}
		  \State Report the falsifying behavior. Exit
		\EndIf
		\EndFor
		\State No falsifying behavior found. Report coverage of $\spec$
\end{algorithmic}
\end{algorithm}
The set $\Gamma$ of timed words of (discrete) length $m$ of the timed automaton $\mathcal{A}$ is generated using the method in \cite{BBBK16}. The function $Solve(\mathcal{O}_{\gamma})$ implements a number of black-box optimization techniques using metaheuristics (such as Simulated Annealing, Evolution Strategies, {\it etc.}), to return a best result $\tilde{\rho}$. This function will be elaborated in Section \ref{sec:combination}.

%To improve search efficiency, that is to quickly find a falsifying input signal, the values $\tilde{\rho}$ of the objective function can be exploited. Note that if an order on $\Gamma$ can be defined, the search can follow a gridding structure as done in \cite{Valko2018} (Valko)... {\color{red} [to elaborate]}


%%% Local Variables:
%%% mode: latex
%%% TeX-master: "main"
%%% End:
