\subsubsection{$\Sigma \Delta$ modulator}
In this section we illustrate the interest of timed pattern generation with a $\Sigma \Delta$ modulator which is an important component of $\Sigma \Delta$ analog-to-digital converters. Such converters have been widely used for analog signals of a large range of frequencies. The most basic architecture of the modulator contains a $1$-bit DAC (comparator), a $1$-bit DAC (switch), one or more integrators. The number of integrators indicates the order of the modulator. The output of the comparator is negatively fed back through the DAC. Unlike many other converters, $\Sigma \Delta$ converters use oversampling which together with decimation filtering and quantization noise shaping allow achieving better anti-aliasing and high resolution. Practical quantizers have a limited input and output ranges, which may lead them to saturation. This constitutes a major non-linearity of the modulator. We apply our methods of signal generation to test if a saturation can occur in a $\Sigma \Delta$ modulator. We use a behavioral model of a second-order modulator specified using Simulink\textsuperscript{\textregistered}, which takes into account most non-idealities \cite{Brigati99}, including sampling jitter, integrator noise, op-amp parameters (finite gain, finite bandwidth, slew-rate and saturation voltages). In terms of model complexity, this Simulink model is heterogeneous including embedded Matlab code and mixing discrete-time and continuous-time components, which goes beyond the applicability of the existing formal verification tools. Simplified discrete-time $\Sigma \Delta$ modulator model without non-idealities, for which it is possible to derive its dynamics equations and thus optimization can be formulated and solved using optimization \cite{DangDM04} and statistical model-checking \cite{ClarkeDL10}.

\begin{figure}[ht]
\resizebox{0.9\textwidth}{!}{
\begin{tikzpicture}[->,>=stealth',shorten >=1pt,auto,node distance=2.1cm, initial text={}]
   \node [accepting,state] (q0)                      {$q_4$};
   \node[accepting,state]          (q1) [right of=q0,yshift=-0.8cm]         {$q_5$};
   \node[accepting,state]          (q2) [left of=q1,yshift=-0.8cm]         {$q_6$};
   \node[accepting,state]          (q3) [left of=q0,yshift=-0.8cm]         {$q_3$};
   \node [accepting,state] (q2bis)    [left of=q3]                  {$q_2$};   
   \node [accepting,state] (q1bis)    [left of=q2bis]                  {$q_1$};
   \node [initial,accepting,state] (q0bis) [left of=q1bis]                      {$q_0$};
   \path (q0) edge [bend left] 
   node {$\begin{array}{c}
         x_1\in (1,6)\\ x_1:=0
   \end{array}$} (q1);
   \path (q1) edge [bend left] 
   node {$\begin{array}{c}
         x_2\in (1,6)\\
        x_2:=0
   \end{array}$} (q2);
   \path (q2) edge [bend left] 
   node {$\begin{array}{c}
         x_3\in (1,6)\\ x_3:=0
   \end{array}$} (q3);
   \path (q3) edge [bend left]  
   node {$\begin{array}{c}
         x_4\in (1,6)\\
        x_4:=0
   \end{array}$} (q0);
   \path (q0bis) edge node [above] {$\begin{array}{c}x_1\in (0,6)\\ x_1:=0\end{array}$} (q1bis);
   \path (q1bis) edge node [above] {$\begin{array}{c}x_2\in (0,6)\\ x_2:=0\end{array}$} (q2bis);
   \path (q2bis) edge node [above] {$\begin{array}{c}x_3\in (0,6)\\ x_3:=0\end{array}$} (q3);
\end{tikzpicture}
}
%\begin{tikzpicture}[->,>=stealth',shorten >=1pt,auto,node distance=1.2cm, initial text={$c=0$}]
%   \node [initial,state,accepting] (q0)                      {$0$};
%      \path (q0) edge [loop above]  node [above]  {$\begin{array}{c} c'=2 c+\lfloor t \rfloor,
%      c'_0=\lfloor t \rfloor,\\ c'_{-1}=c_0, c'_{-2}=c_{-1}, c'_{-3}=c_{-2},\\
%      \textcolor{purple}{c_0+c_{-1}+c_{-2}+c_{-3} \in\{1,2\}}\end{array}$} (q0);
%\end{tikzpicture}

\caption{A timed automaton for the period constraint.}
 \label{fig:4.5clocks}
\end{figure}
%(see Fig.~\ref{fig:DeltaSigma}),
%\begin{figure}[!ht]
%  \centering
%  \includegraphics[width=0.9\textwidth]{figures/DSfig.pdf}
%  \caption{$\Sigma \Delta$ model with non-idealities \cite{Brigati99}.\label{fig:DeltaSigma}}
%\end{figure}%\vspace{-1cm}
As input signals, we consider a class of quasi-periodic signals with uncertain period duration that ranges between $1$ and $6$. The period duration can be scaled to consider the frequency range of interest. The temporal pattern of the considered signals is specified by a timed automaton in Figure \ref{fig:4.5clocks}. We discretise the signal value space and associate each transition label with a signal value range. The transitions should reflect some bounded variability condition. Then the signals are constructed by linear interpolation between the values at time instants of the timed words. The absence of saturation can be expressed as a simple STL property, which states that throughout the simulation duration from $0$ to $T$, the absolute value of the output of the saturation block in the first integrator is always smaller than the saturation voltage value (which is $2$).

We generate a set of $100$ timed words using uniform sampling \cite{} {\color{red}TODO}. For the optimization part, we use Algorithm \ref{algoSolverCombination} for guided  combination of metaheuristics. 
This experiment allows observing that combining optimization with timed pattern generation allowed us to falsify the absence of saturation after using 22 generated timed patterns, while {\color{red}TODO}