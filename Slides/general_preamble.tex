\usepackage{booktabs}
\usepackage[scale=2]{ccicons}
%\usepackage{minted}

\usepackage{pgfplots}
\usepgfplotslibrary{dateplot}
%\usemintedstyle{trac}

%%%%%%%%%%%%%%%%%%%%%%%%%%%%%%%%%%%%%%%
%%%%%%%%%%%%%%%%%%%%%%%%%%%%%%%%%%%%%%%
%%%%%%%%%%%%%%%%%%%%%%%%%%%%%%%%%%%%%%%
% \usepackage{sansmath} % doesn't seem to make a difference
%\usepackage{mathptmx}
%\usepackage{newtxsf}
% Using standard mathcal letters
%\DeclareMathAlphabet{\mathcal}{OMS}{cmsy}{m}{n}

%%%%%%%%%%%%%%%%%%%%%%%%%%%%%%%%%%%%%%%
\usepackage{graphicx}        % standard LaTeX graphics tool
                             % when including figure files
%\usepackage{multicol}        % used for the two-column index
%\usepackage[bottom]{footmisc}% places footnotes at page bottom
\usepackage{url}

%\usepackage{mchchapters}

\usepackage{amssymb,amsmath}

% smaller underbrace
\makeatletter
\def\smallunderbrace#1{\mathop{\vtop{\m@th\ialign{##\crcr
   $\hfil\displaystyle{#1}\hfil$\crcr
   \noalign{\kern3\p@\nointerlineskip}%
   \tiny\upbracefill\crcr\noalign{\kern3\p@}}}}\limits}
\makeatother
\newcommand{\clap}[1]{\makebox[0pt]{#1}}

\usepackage{stmaryrd,latexsym}
\usepackage{units} % for nicefrac

\usepackage{tikz}
\usetikzlibrary{positioning}
\usetikzlibrary{calc}
\usetikzlibrary{arrows,automata}
\usetikzlibrary{decorations.pathreplacing,shapes.misc}
\usetikzlibrary{decorations.markings}

\usepackage[ruled, vlined]{algorithm2e}
%\usepackage{gastex}
%\usepackage{epic,eepic}      % macros for latex-figures exported from Xfig
%\usepackage{comment}

% GF 2014-12-15 >>>>
\usepackage{xfrac}
\usepackage{mathtools} % for column vector
% note: not allowed \usepackage{subcaption}
\usepackage{subfigure}
%\usepackage[svgnames]{xcolor} % for tikz
\usepackage{framed} % for to do, commments etc.
\usepackage{xcolor}  % for comments
%\definecolor{darkgreen}{rgb}{0,0.3,0}
\definecolor{darkorange}{rgb}{0.9,0.5,0.1}
\definecolor{darkgreen}{rgb}{0.25,0.5,0.25}
\definecolor{lightgreen}{rgb}{0.925,1,0.875}
\definecolor{gray}{rgb}{0.8,0.8,0.8} % use lighter gray for presentation
%\usepackage{etex} % required for using pgfplots
\usepackage{pgfplots}
\usepackage{pgfplotstable} % used for ODE plots
\pgfplotsset{compat=1.14}
\usetikzlibrary{fpu} % for higher accuracy
\usepackage{pgfplotstable} % for plotting functions

\pgfplotsset{major grid style={dotted,gray}}

% <<<< GF
%%%%%%%%%%%%%%%%%%%%%%%%%%%%%%%%%%%%%%%
%%%%%%%%%%%%%%%%%%%%%%%%%%%%%%%%%%%%%%%
%%%%%%%%%%%%%%%%%%%%%%%%%%%%%%%%%%%%%%%


%\let\emptyset\varnothing
%\def\break{\penalty-1000}
%
%
%\def\frp#1{\ensuremath{\langle #1\rangle}}
%
%\makeatletter
%\let\c@lemm\c@theo
%\let\c@coro\c@theo
%\let\c@defi\c@theo
%\let\c@assu\c@theo
%\makeatother
%
%\usepackage{prettyref}
%\let\rref\prettyref
%\newrefformat{part}{Part\,\ref{#1}}
%\newrefformat{ch}{Chap.\,\ref{#1}}
%\newrefformat{sec}{Sect.\,\ref{#1}}
%\newrefformat{app}{App.\,\ref{#1}}
%\newrefformat{def}{Definition\,\ref{#1}}
%\newrefformat{thm}{Theorem\,\ref{#1}}
%\newrefformat{prop}{Proposition\,\ref{#1}}
%\newrefformat{lem}{Lemma\,\ref{#1}}
%\newrefformat{cor}{Corollary\,\ref{#1}}
%\newrefformat{ex}{Example\,\ref{#1}}
%\newrefformat{cex}{Counterexample\,\ref{#1}}
%\newrefformat{tab}{Table\,\ref{#1}}
%\newrefformat{fig}{Fig.\,\ref{#1}}
%\newrefformat{calculus}{Fig.\,\ref{#1}}
%\newrefformat{case}{Case\,\ref{#1}}
%\newrefformat{line}{line\,\ref{#1}}
%%
%%
%\def\sectionname{Section}
%\def\definitionname{Definition}
%\def\propositionname{Proposition}
%\def\equationname{Equation}
%\def\algorithmname{Algorithm}
%\def\theoremname{Theorem}

%\newcommand{\code}[1]{\texttt{#1}}
\newcommand{\code}[1]{{#1}}

\newcommand{\dfn}[2][]{\alert{#2}}
\providecommand{\m}[1]{\mbox{$#1$}}
\newcommand{\inv}{F}
\newcommand{\ivr}{\Inv}
\newcommand{\precond}{\Init}
\newcommand{\postcond}{\Safe}
\newcommand{\der}[2][]{\nabla_{#1}#2}
\newcommand{\D}[2][]{\dot{#2}}
\newcommand{\Dp}[2][]{\frac{\partial #2}{\partial #1}}
\newcommand{\Dt}[2][]{\frac{d #2}{d #1}}
\newcommand{\dL}{\text{\upshape\textsf{d{\kern-0.1em}$\mathcal{L}$}}\xspace}
\newcommand{\KeYmaera}{KeYmaera\xspace}%{Ke\kern-0.1emYmaera\xspace}

    \newcommand{\ltrue}{\mathit{true}}
    \newcommand{\lfalse}{\mathit{false}}
    \let\limply\rightarrow
    \let\lreverseimply\leftarrow
    \let\lylpmi\lreverseImply
    \let\lbisubjunct\leftrightarrow
\let\landfold\bigwedge
\let\lorfold\bigvee
\newcommand{\lforall}[3][]{\lquantifier{\forall}{#1}{#2}{#3}}
\newcommand{\lexists}[3][]{\lquantifier{\exists}{#1}{#2}{#3}}
\newcommand{\lquantifier}[4]
           {#1%
           #3{\,}
           #4}

\newcommand{\dbox}[2]{[#1]#2}
\newcommand{\ddiamond}[2]{\langle#1\rangle#2}
\newcommand{\prepeat}[1]{#1^*}
\newcommand{\hevolve}[1]{#1}
\newcommand{\hevolvein}[2]{#1\&#2}
\newcommand{\syssep}{,}
\newcommand{\bigo}{\mathcal{O}}


% ------------------------------------------    Automata     ----------------------------------------------------

\newcommand{\HAvars}[1]{(\Loc, \Lab,\break \Edg,\break #1,\break \Init, \Inv, \Flow, \Jump)}
\newcommand{\HA}{\HAvars{X}}
\newcommand{\CS}{\tuple{X,\break \Init, \Inv, \Flow}}
\newcommand{\PCDA}{PCDA\xspace}
\newcommand{\PADA}{PWA\xspace}

\newcommand{\Loc}{{\sf Loc}} 
\newcommand{\loc}{{loc}} 
%\newcommand{\Var}{{\sf Var}}
%\newcommand{\Clock}{{\sf C}} 
%\newcommand{\Param}{{\sf P}}
\newcommand{\Lab}{{\sf Lab}} 
\newcommand{\Edg}{{\sf Edg}} 

\newcommand{\Init}{{\sf Init}} 
\newcommand{\Inv}{{\sf Inv}} 
\newcommand{\Flow}{{\sf Flow}}
\newcommand{\Jump}{{\sf Jump}}
\newcommand{\Final}{{\sf Final}} 

\newcommand{\silent}{\tau} 

\newcommand{\location}[1]{\ensuremath{\mathit{#1}}}

% ------------------------------------------    Clock constraints     ----------------------------------------------------

\newcommand{\Rect}{{\sf Rect}}
\newcommand{\Rectc}{{\sf Rect_c}}
\newcommand{\Recto}{{\sf Rect_o}}
\newcommand{\Lin}{{\sf LConstr}}
\newcommand{\Term}{{\sf LTerm}}
\newcommand{\PTerm}{{\sf PTerm}}
\newcommand{\Poly}{{\sf PConstr}}
\newcommand{\Affine}{{\sf Affine}}

\newcommand{\MultRect}{{\sf MultRect}}
\newcommand{\PRect}{{\sf PRect}}
\newcommand{\PRectc}{{\sf PRect_c}}
\newcommand{\MultPRect}{{\sf MultPRect}}

% ------------------------------------------    Infix operators     -------------------------------------------------



\def\sg{\llbracket}
\def\sd{\rrbracket}

\newcommand{\sem}[1]{\sg #1 \sd} 
\newcommand{\tuple}[1]{\langle #1 \rangle}

\def\abs#1{\ensuremath{\lvert #1\rvert}}
\def\bigabs#1{\ensuremath{\big\lvert #1 \big\rvert}}
\def\norm#1{\ensuremath{\lVert #1\rVert}}
\def\inftynorm#1{\ensuremath{\lVert #1\rVert_\infty}}

%\newcommand{\timeSucc}[2]{#1\!\rotatebox{15}{$\nearrow$}\!#2}
\newcommand{\timeSucc}[2]{#1\!\nearrow\!#2}
% Time successor for strictly positive time
\newcommand{\timeSuccSPos}[2]{\timeSucc{#1}{\!\!\!\!_{>0}\;#2}}
\newcommand{\timeSuccCl}[2]{\timeSucc{#1}{\!\!\!\!_{\geq 0 }\;#2}}

% --- Matrix ---

\newcommand{\mx}[1]{#1}%{\mathbf{#1}}


% ------------------------------------------    Number Sets     ----------------------------------------------------

\newcommand{\Bool}{\mathbb B} 
\newcommand{\rat}{{\mathbb Q}}
\newcommand{\ratl}{{\mathbb L}}
\newcommand{\nat}{\mathbb N} 
\let\naturals\nat
\newcommand{\sposrat}{{\mathbb Q}^{> 0}}
\newcommand{\posrat}{{\mathbb Q}^{\geq 0}}
\newcommand{\posreal}{{\mathbb R}^{\geq 0}}
\newcommand{\sposreal}{{\mathbb R}^{> 0}}
\newcommand{\real}{{\mathbb R}}
\let\reals\real
\newcommand{\realsnneg}{{\mathbb R}^{\geq 0}}
\newcommand{\zed}{{\mathbb Z}}


% ------------------------------------------    Font commands     ----------------------------------------------------

\newcommand{\ceil}[1]{\left \lceil #1 \right \rceil }
\newcommand{\floor}[1]{\left \lfloor #1 \right \rfloor }

\renewcommand{\l}{{\ell}}

\newcommand{\I}{\mathcal{I}}
\newcommand{\conv}{{\sf conv}}


\newcommand{\bad}{\mathsf{bad}}
\newcommand{\post}{{\sf post}}
\newcommand{\Post}{{\sf Post}}
\newcommand{\Pre}{{\sf Pre}}
\newcommand{\pre}{{\sf pre}}
\newcommand{\Reach}{{\sf Reach}}
\newcommand{\Unsafe}{{\sf Unsafe}} 
\newcommand{\Safe}{\mathsf{Safe}}
\newcommand{\update}{{\sf update}}
\newcommand{\Verify}{{\sc Verify}}


\renewcommand{\time}{{\sf time}}
\newcommand{\EN}{{\sf EN}}


\newcommand{\true}{{\sf true}}
\newcommand{\false}{{\sf false}}


%\newcommand{\TTS}{{\sf TTS}} 

\newcommand{\hytech}{{\sc HyTech}}
\newcommand{\uppaal}{{\sc Uppaal}}
\newcommand{\kronos}{{\sc Kronos}}
\newcommand{\phaver}{{\sc PHAVer}}
\newcommand{\ddt}{{\sc d}/{\sc dt}} % GF: 2014-03-04 changed typesetting to be consistent with other tools
\newcommand{\Giotto}{{\sc Giotto}}
\newcommand{\Charon}{{\sc Charon}}
\newcommand{\Esterel}{{\sc Esterel}}
\newcommand{\Lustre}{{\sc Lustre}}
\newcommand{\Times}{{\sc Times}}
\newcommand{\rabbit}{{\sc Rabbit}}


%\def\mynote#1{{\sf $\clubsuit$ #1 $\clubsuit$}}
\def\mynote#1{}

%%%%%%%%%%%%%%%%%%%%%%%%%%%%%%%%%%%%%%%%
% GF 2013-12-15 >>>>
\newcounter{todocounter}
\newcommand{\todo}[2][Author]{
\par\noindent\parbox{\linewidth}{
\color{blue}
\begin{framed} 
\noindent{\sffamily{\bfseries To do 
\stepcounter{questioncounter}\thequestioncounter
} (#1):  
#2
} 
\end{framed} 
\color{black} %\bigskip 
}
}
\newcounter{questioncounter}
\newcommand{\myquestion}[2][Author]{
\par\noindent\parbox{\linewidth}{
\color{red}
\begin{framed} 
\noindent{\sffamily{\bfseries Question 
\stepcounter{questioncounter}\thequestioncounter
} (#1):  
#2
} 
\end{framed} 
\color{black} %\bigskip 
}
}
\newcounter{commentcounter}
\newcommand{\mycomment}[2][Author]{\protect\color{blue}
% GF: somehow framed causes odd errors (TeX capacity exceeded)
%\begin{framed} 
%\bigskip
\par
\noindent \parbox{\linewidth}{
\noindent{\sffamily{\bfseries Comment \stepcounter{commentcounter}\thecommentcounter} (#1):  #2}
}
 %\end{framed} 
%\bigskip
\par
 \protect\color{black} %\bigskip 
}
\renewcommand{\myquestion}[2][Author]{}
\renewcommand{\mycomment}[2][Author]{}

\newcommand{\myparagraph}[1]{\bigskip\par\noindent{\bfseries #1.}}


%%%%%%
% Compute angles in Tikz
\newcommand{\tikzAngleOfLine}{\tikz@AngleOfLine}
\def\tikz@AngleOfLine(#1)(#2)#3{%
  \pgfmathanglebetweenpoints{%
    \pgfpointanchor{#1}{center}}{%
    \pgfpointanchor{#2}{center}}
  \pgfmathsetmacro{#3}{\pgfmathresult}%
}
%%%%%

\def \transp {^{\mathsf{T}}\!}
\newcommand{\linconjunctive}{conjunctive\xspace}
%\newcommand{\colvec}[1]{\left\[\begin{pmatrix}#1\end{pmatrix}\right\]}
\newcommand{\colvec}[1]{\begin{pmatrix}#1\end{pmatrix}}
\newcommand{\smallcolvec}[1]{\begin{psmallmatrix}#1\end{psmallmatrix}}
\newcommand{\Pos}[1]{\mathrm{pos}(#1)}
\def \Chull {\mathrm{chull}}
%\def \supfun {\varrho}
\newcommand{\supfun}[2]{\ensuremath{\rho_{#1}({#2})}} % support function{set,direction}
\newcommand{\outerappr}[1]{\left\lceil#1\right\rceil}

% use bold vectors
\renewcommand{\vec}[1]{\mathbf{#1}}
\def \x {\ensuremath{\vec{x}}\xspace}
\def \y {\ensuremath{\vec{y}}\xspace}
\def \z {\ensuremath{\vec{z}}\xspace}
\def \q {\ensuremath{\vec{q}}\xspace}
\def \p {\ensuremath{\vec{p}}\xspace}
\def \b {\ensuremath{\vec{b}}\xspace}
\def \d {\ensuremath{\vec{d}}\xspace}
\def \u {\ensuremath{\vec{u}}\xspace}
\def \v {\ensuremath{\vec{v}}\xspace}
\def \w {\ensuremath{\vec{w}}\xspace}
\def \U {\ensuremath{\mathcal{U}}\xspace}
\def \W {\ensuremath{\mathcal{W}}\xspace}
\def \X {\ensuremath{\mathcal{X}}\xspace}
\def \Y {\ensuremath{\mathcal{Y}}\xspace}
\def \Z {\ensuremath{\mathcal{Z}}\xspace}
\def \U {\ensuremath{\mathcal{U}}\xspace}
\def \V {\ensuremath{\mathcal{V}}\xspace}
\def \W {\ensuremath{\mathcal{W}}\xspace}
\def \B {\ensuremath{\mathcal{B}}\xspace}
\def \C {\ensuremath{\mathcal{C}}\xspace}
\def \D {\ensuremath{\mathcal{D}}\xspace}
\def \E {\ensuremath{\mathcal{E}}\xspace}
\def \F {\ensuremath{\mathcal{F}}\xspace}
\def \G {\ensuremath{\mathcal{G}}\xspace}
\def \H {\ensuremath{\mathcal{H}}\xspace}
\def \I {\ensuremath{\mathcal{I}}\xspace}
\def \N {\ensuremath{\mathcal{N}}\xspace}
\def \R {\ensuremath{\mathcal{R}}\xspace}
\def \P {\ensuremath{\mathcal{P}}\xspace}
\def \Q {\ensuremath{\mathcal{Q}}\xspace}
\def \Sset {\ensuremath{\mathcal{S}}\xspace} % note that \S is already defined
\def \Approx {\ensuremath{\mathrm{Appr}}}
\def \unitvec{\ensuremath{\hat{\vec{e}}}}
% <<<< GF
%%%%%%%%%%%%%%%%%%%%%%%%%%%%%%%%%%%%%%%%



%%%%%%%%%%%%%%%%%%%%%%%%%%%%%%%%%%%%%%%%
% Hyphenations
\hyphenation{PCDA}
\hyphenation{PADA}
%%%%%%%%%%%%%%%%%%%%%%%%%%%%%%%%%%%%%%%%
